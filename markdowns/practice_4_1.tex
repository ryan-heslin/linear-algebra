\PassOptionsToPackage{unicode=true}{hyperref} % options for packages loaded elsewhere
\PassOptionsToPackage{hyphens}{url}
%
\documentclass[]{article}
\usepackage{lmodern}
\usepackage{amssymb,amsmath}
\usepackage{ifxetex,ifluatex}
\usepackage{fixltx2e} % provides \textsubscript
\ifnum 0\ifxetex 1\fi\ifluatex 1\fi=0 % if pdftex
  \usepackage[T1]{fontenc}
  \usepackage[utf8]{inputenc}
  \usepackage{textcomp} % provides euro and other symbols
\else % if luatex or xelatex
  \usepackage{unicode-math}
  \defaultfontfeatures{Ligatures=TeX,Scale=MatchLowercase}
\fi
% use upquote if available, for straight quotes in verbatim environments
\IfFileExists{upquote.sty}{\usepackage{upquote}}{}
% use microtype if available
\IfFileExists{microtype.sty}{%
\usepackage[]{microtype}
\UseMicrotypeSet[protrusion]{basicmath} % disable protrusion for tt fonts
}{}
\IfFileExists{parskip.sty}{%
\usepackage{parskip}
}{% else
\setlength{\parindent}{0pt}
\setlength{\parskip}{6pt plus 2pt minus 1pt}
}
\usepackage{hyperref}
\hypersetup{
            pdftitle={Section 4.1 Problems},
            pdfauthor={Ryan Heslin},
            pdfborder={0 0 0},
            breaklinks=true}
\urlstyle{same}  % don't use monospace font for urls
\usepackage[margin=1in]{geometry}
\usepackage{color}
\usepackage{fancyvrb}
\newcommand{\VerbBar}{|}
\newcommand{\VERB}{\Verb[commandchars=\\\{\}]}
\DefineVerbatimEnvironment{Highlighting}{Verbatim}{commandchars=\\\{\}}
% Add ',fontsize=\small' for more characters per line
\usepackage{framed}
\definecolor{shadecolor}{RGB}{248,248,248}
\newenvironment{Shaded}{\begin{snugshade}}{\end{snugshade}}
\newcommand{\AlertTok}[1]{\textcolor[rgb]{0.94,0.16,0.16}{#1}}
\newcommand{\AnnotationTok}[1]{\textcolor[rgb]{0.56,0.35,0.01}{\textbf{\textit{#1}}}}
\newcommand{\AttributeTok}[1]{\textcolor[rgb]{0.77,0.63,0.00}{#1}}
\newcommand{\BaseNTok}[1]{\textcolor[rgb]{0.00,0.00,0.81}{#1}}
\newcommand{\BuiltInTok}[1]{#1}
\newcommand{\CharTok}[1]{\textcolor[rgb]{0.31,0.60,0.02}{#1}}
\newcommand{\CommentTok}[1]{\textcolor[rgb]{0.56,0.35,0.01}{\textit{#1}}}
\newcommand{\CommentVarTok}[1]{\textcolor[rgb]{0.56,0.35,0.01}{\textbf{\textit{#1}}}}
\newcommand{\ConstantTok}[1]{\textcolor[rgb]{0.00,0.00,0.00}{#1}}
\newcommand{\ControlFlowTok}[1]{\textcolor[rgb]{0.13,0.29,0.53}{\textbf{#1}}}
\newcommand{\DataTypeTok}[1]{\textcolor[rgb]{0.13,0.29,0.53}{#1}}
\newcommand{\DecValTok}[1]{\textcolor[rgb]{0.00,0.00,0.81}{#1}}
\newcommand{\DocumentationTok}[1]{\textcolor[rgb]{0.56,0.35,0.01}{\textbf{\textit{#1}}}}
\newcommand{\ErrorTok}[1]{\textcolor[rgb]{0.64,0.00,0.00}{\textbf{#1}}}
\newcommand{\ExtensionTok}[1]{#1}
\newcommand{\FloatTok}[1]{\textcolor[rgb]{0.00,0.00,0.81}{#1}}
\newcommand{\FunctionTok}[1]{\textcolor[rgb]{0.00,0.00,0.00}{#1}}
\newcommand{\ImportTok}[1]{#1}
\newcommand{\InformationTok}[1]{\textcolor[rgb]{0.56,0.35,0.01}{\textbf{\textit{#1}}}}
\newcommand{\KeywordTok}[1]{\textcolor[rgb]{0.13,0.29,0.53}{\textbf{#1}}}
\newcommand{\NormalTok}[1]{#1}
\newcommand{\OperatorTok}[1]{\textcolor[rgb]{0.81,0.36,0.00}{\textbf{#1}}}
\newcommand{\OtherTok}[1]{\textcolor[rgb]{0.56,0.35,0.01}{#1}}
\newcommand{\PreprocessorTok}[1]{\textcolor[rgb]{0.56,0.35,0.01}{\textit{#1}}}
\newcommand{\RegionMarkerTok}[1]{#1}
\newcommand{\SpecialCharTok}[1]{\textcolor[rgb]{0.00,0.00,0.00}{#1}}
\newcommand{\SpecialStringTok}[1]{\textcolor[rgb]{0.31,0.60,0.02}{#1}}
\newcommand{\StringTok}[1]{\textcolor[rgb]{0.31,0.60,0.02}{#1}}
\newcommand{\VariableTok}[1]{\textcolor[rgb]{0.00,0.00,0.00}{#1}}
\newcommand{\VerbatimStringTok}[1]{\textcolor[rgb]{0.31,0.60,0.02}{#1}}
\newcommand{\WarningTok}[1]{\textcolor[rgb]{0.56,0.35,0.01}{\textbf{\textit{#1}}}}
\usepackage{graphicx,grffile}
\makeatletter
\def\maxwidth{\ifdim\Gin@nat@width>\linewidth\linewidth\else\Gin@nat@width\fi}
\def\maxheight{\ifdim\Gin@nat@height>\textheight\textheight\else\Gin@nat@height\fi}
\makeatother
% Scale images if necessary, so that they will not overflow the page
% margins by default, and it is still possible to overwrite the defaults
% using explicit options in \includegraphics[width, height, ...]{}
\setkeys{Gin}{width=\maxwidth,height=\maxheight,keepaspectratio}
\setlength{\emergencystretch}{3em}  % prevent overfull lines
\providecommand{\tightlist}{%
  \setlength{\itemsep}{0pt}\setlength{\parskip}{0pt}}
\setcounter{secnumdepth}{0}
% Redefines (sub)paragraphs to behave more like sections
\ifx\paragraph\undefined\else
\let\oldparagraph\paragraph
\renewcommand{\paragraph}[1]{\oldparagraph{#1}\mbox{}}
\fi
\ifx\subparagraph\undefined\else
\let\oldsubparagraph\subparagraph
\renewcommand{\subparagraph}[1]{\oldsubparagraph{#1}\mbox{}}
\fi

% set default figure placement to htbp
\makeatletter
\def\fps@figure{htbp}
\makeatother


\title{Section 4.1 Problems}
\author{Ryan Heslin}
\date{2021-11-17}

\begin{document}
\maketitle

\% Standard custom LaTeX commands
\newcommand{\abcd}{\begin{bmatrix}a&b\\
c&d\end{bmatrix}}

\newcommand{\m}[1]{\begin{bmatrix}#1\end{bmatrix}}

\newcommand{\vect}[1]{\begin{pmatrix}#1\end{pmatrix}}

\newcommand{\meq}[1]{\begin{split}#1\end{split}}

\newcommand{\bym}[1]{#1\times{m}}

\newcommand{\nby}[1]{n\times{#1}}

\%

\newcommand{\subsp}[2]{\Bigg\{\begin{bmatrix}#1\end{bmatrix}:#2\Bigg\}}

\% 1: term 1 \% 2: subscript 1 \% 3: term 2 \% 4: subscript 2 \% 5.
operation

\newcommand{\dotsn}[5]{#1_2#3_1#5#1_2#3_2#5#1_{#2}#3_{#4}}

Check some subspaces

\hypertarget{section}{%
\section{1.}\label{section}}

\[p(0)=2\]

Not closed under addition:

\[
ap(0)+bp(0)=2a+2b \neq 2
\]

\hypertarget{section-1}{%
\section{2.}\label{section-1}}

\[p(2)=0\]

Nice and closed. \[ap(2) +bp(2) = 0a + 0a = 0\]

\[k(p2)=0k=0\]

\hypertarget{section-2}{%
\section{3.}\label{section-2}}

\[\begin{aligned}& (f(x)+g(x))'=f'(x)+g'(x)\\
& 1 + 1 = f(x) + g(x)\end{aligned}\]

\[\begin{aligned} & 4a + 2b +c + 4d + 2e +f\\
&4(a+d)+2(b + e) +(f+ c)\end{aligned}\]

\hypertarget{section-3}{%
\section{4.}\label{section-3}}

Valid subspace \[
  \begin{aligned}
    & \int_0^1(p(t)dt + \int_0^1g(h)dh=0 + 0 = 0\\
    & k\int_0^1p(t)dt =k0 =0
  \end{aligned}
\]

Basis: \((1, t, t^2)\)

\hypertarget{section-4}{%
\section{5.}\label{section-4}}

\(p(-t)=-p(t)\). This satisfies the scalar axiom by definition.

\hypertarget{section-5}{%
\section{6.}\label{section-5}}

\(3\times{3}\) invertibles are not a subspace because not closed under
addition:

\begin{Shaded}
\begin{Highlighting}[]
\KeywordTok{library}\NormalTok{(matador, }\DataTypeTok{quietly =} \OtherTok{TRUE}\NormalTok{)}
\KeywordTok{try}\NormalTok{(}\KeywordTok{solve}\NormalTok{(}\KeywordTok{diag}\NormalTok{(}\DataTypeTok{nrow =} \DecValTok{3}\NormalTok{) }\OperatorTok{+}\StringTok{ }\KeywordTok{square}\NormalTok{(}\DecValTok{0}\NormalTok{, }\DecValTok{1}\NormalTok{, }\DecValTok{1}\NormalTok{, }\DecValTok{1}\NormalTok{, }\DecValTok{0}\NormalTok{, }\DecValTok{1}\NormalTok{,}
    \DecValTok{1}\NormalTok{, }\DecValTok{1}\NormalTok{, }\DecValTok{0}\NormalTok{)))}
\end{Highlighting}
\end{Shaded}

\begin{verbatim}
Error in solve.default(diag(nrow = 3) + square(0, 1, 1, 1, 0, 1, 1, 1,  : 
  Lapack routine dgesv: system is exactly singular: U[2,2] = 0
\end{verbatim}

\hypertarget{section-6}{%
\section{7.}\label{section-6}}

Diagonals are obviously a subspace.

\hypertarget{section-7}{%
\section{8.}\label{section-7}}

Ditto upper triangular; the zero elements never become nonzero.

\hypertarget{section-8}{%
\section{9.}\label{section-8}}

\(3\times{3}\) with positive nonzero entries: yes.

\hypertarget{section-9}{%
\section{10}\label{section-9}}

Matrices whose kernel is \(v=\begin{bmatrix}1\\2\\3\end{bmatrix}\): yes.
We can use the properties since all matrix transformations are linear.

\[\begin{aligned}&Av=0&Bv=0\\
&Av+Bv = (A+B)v=0\\
&kAv=0\\
&A(kv)=0\end{aligned}\]

\hypertarget{section-10}{%
\section{11.}\label{section-10}}

\(3\times{3}\) RREFS: not closed under scalar multiplication or
addition, since scaling converts to non-RREF form.

The following concern the space of infinite sequences.

\hypertarget{section-11}{%
\section{12.}\label{section-11}}

\((a, a+k, a+2k,\dots)\) is a subspace.

\[\begin{aligned}&A =(a, a+k, a+2k,\dots)\\
& B= (b, b+c, b+2c,\dots)\\
&A+B=(a+b)+(a+b+k+c) +(a+b+2k+2c)\\
&cA=(ca+(ca+ck)+ca+2kc)=(ca+(ca+ck)+ca+2kc)\end{aligned}\]

\hypertarget{section-12}{%
\section{13.}\label{section-12}}

Geometric sequences \((a, ar, ar^2, ar^3,\dots)\) are not a subspace.

Not closed under addition:

\[\begin{aligned}
&(a,ar,ar^2)+(b, bq, bq^2)=((a+b), (a+b)(r+q), (a+b)(r^2+q^2)\\
&(a+b, ar+bq, ar^2+bq^2) \neq((a+b), ar+br+aq+bq, ar^2+br^2+aq^2+bq^2
\end{aligned}\]

Scalar multiplication

\[\begin{aligned}&k(a, ar, ar^2)=(ka, kar, kar^2)\\
&(ka, kar, kar^2)=(ka, kar, kar^2)\end{aligned}\]

\hypertarget{section-13}{%
\section{14.}\label{section-13}}

Sequences that converge on 0 are a subspace, because limits obey the
adding and scalar multiplication axioms.

\[\begin{aligned} &A+B=0\\
& k(A)=0=A(k)=0\end{aligned}\]

\hypertarget{section-14}{%
\section{15.}\label{section-14}}

Square-summable (converge on \(\sum^{\infty}_{i=0}x^2_i\) are not a
subspace. The squares of the summed sequence are not the same as those
of the separate sequences.

\[\begin{aligned}
&X +Y = (x_1+y_1), (x_2+y_2),\dots,(x_n+y_n)\\
&\sum^{\infty}_{i=0} = (x^2+y^2+2xy,\dots,x_n^2+y_n^2+2x_ny_n)\end{aligned}\]

Now we find bases.

\hypertarget{section-15}{%
\section{16.}\label{section-15}}

\(R^{3\times{2}}\): one-hot matrices with a 1 in each of the six
elements, dimension 6.

\hypertarget{section-16}{%
\section{17.}\label{section-16}}

\(R^{n \times {m}}\): \(mn\) one-hot matrices, dimension \(mn\).

\hypertarget{section-17}{%
\section{18.}\label{section-17}}

All \(2\times{2}\) with trace that sums to 0:

\[\begin{bmatrix}1&0\\
0&0\end{bmatrix},
\begin{bmatrix}0&b\\
0&0\end{bmatrix},
\begin{bmatrix}0&0\\
c&0\end{bmatrix}\]

We don't need a basis for \(d\) because it's the opposite sign of \(a\).
So we've lost one degree of freedom.

\hypertarget{section-18}{%
\section{19.}\label{section-18}}

\(C^2\): \((1, i)\)

\hypertarget{section-19}{%
\section{20.}\label{section-19}}

All diagonal matrices: \(n\) one-hot matrices, one with 1 in each
diagonal position

\hypertarget{section-20}{%
\section{24.}\label{section-20}}

Lower and upper triangular matrices: standard one-hots, with dimension
\(\sum_{i=1}^n\).

\hypertarget{section-21}{%
\section{25.}\label{section-21}}

All polynomials \(P_2\) such that \(f(1)=0\): dimension is 2, since
\(a +1b + 1c=0\). A basis could be \(1, t\).

Not at all right!

\hypertarget{section-22}{%
\section{26.}\label{section-22}}

\hypertarget{section-23}{%
\section{27.}\label{section-23}}

Such a matrix implies the system:

\[\begin{aligned}&a^2+bc=1\\
&ab+dc=0\\
&ac+bd=0\\
&bc+d^2=2\end{aligned}\]

which requires the off-diagonal to be 0. So the components matrix itself
are the basis

\[\begin{bmatrix}1&0\\0&1\end{bmatrix}\]

Note the basis doesn't have the ratio of components to each other of the
final matrix: keep it one hot.

\hypertarget{section-24}{%
\section{28.}\label{section-24}}

\hypertarget{section-25}{%
\section{29.}\label{section-25}}

\[
   a \begin{\bmatrix}
    1 & -1 \\
    0  & 0
    \end{\bmatrix} +
    b \begin{\bmatrix}
    0 & 0\\
    1 & -1
    \end{\bmatrix}          
\]

\hypertarget{section-26}{%
\section{30.}\label{section-26}}

\[\begin{bmatrix}1&2\\3&6\end{bmatrix}A=\begin{bmatrix}0&0\\
0&0\end{bmatrix}\]

Dimension 2. Rows have to be multiples of \((1, -3)\)

\[\begin{bmatrix}1&0\\
1&0\end{bmatrix}\]

\hypertarget{section-27}{%
\section{31.}\label{section-27}}

\[\begin{bmatrix}0&0\\1&0\end{bmatrix},\begin{bmatrix}0&-1\\0&0\end{bmatrix}\]

dimension 2.

\hypertarget{section-28}{%
\section{32.}\label{section-28}}

\[
    a  \begin{\bmatrix}
    1 & 0\\
    1 & 0
    \end{\bmatrix} + b 
    \begin{\bmatrix}
    0 & 1\\
    0 & -1
    \end{\bmatrix}
\]

\hypertarget{section-29}{%
\section{33.}\label{section-29}}

Just the zero matrix (disclosure: initially wrong).

\hypertarget{section-30}{%
\section{34.}\label{section-30}}

\begin{Shaded}
\begin{Highlighting}[]
\KeywordTok{solve}\NormalTok{(}\KeywordTok{matrix}\NormalTok{(}\KeywordTok{c}\NormalTok{(}\DecValTok{3}\NormalTok{, }\DecValTok{4}\NormalTok{, }\DecValTok{2}\NormalTok{, }\DecValTok{5}\NormalTok{), }\DataTypeTok{nrow =} \DecValTok{2}\NormalTok{))}
\end{Highlighting}
\end{Shaded}

\begin{verbatim}
           [,1]       [,2]
[1,]  0.7142857 -0.2857143
[2,] -0.5714286  0.4285714
\end{verbatim}

Find \(S\) for \(\begin{bmatrix}3&2\\4&5\end{bmatrix}S=S\).

The implied system:

\[\begin{aligned}&a=3a+2c\\
&b=3b+2d\\
&c = 4a+5c\\
&d=4b+5d
\end{aligned}\]

Thus:

\[\begin{aligned}& a=-c\\
& b=-d
\end{aligned}\]

So a basis is
\(\begin{bmatrix}1&0\\0&1\end{bmatrix},\begin{bmatrix}0&1\\0&0\end{bmatrix}\)

\hypertarget{section-31}{%
\section{35.}\label{section-31}}

Diagonal matrices

\hypertarget{section-32}{%
\section{36.}\label{section-32}}

Again, diagonals.

\hypertarget{section-33}{%
\section{37.}\label{section-33}}

Diagonal matrices, so {[}0, 3{]}

\hypertarget{section-34}{%
\section{38.}\label{section-34}}

{[}0, 4{]}.

\hypertarget{section-35}{%
\section{39.}\label{section-35}}

The dimension of the space of all upper triangular is \(\sum^n_{i=1}\).
For a \(3\times{3}\) it is 6. Better, \(n \choose 2\).

\hypertarget{section-36}{%
\section{40.}\label{section-36}}

\(n^2-n,n^2-2n,\dots,0\). Each increment of rank adds \(n\) elements to
the basis for the matrix. If \(c\) is the zero vector, the dimension
could b3 \(n^2\), since full-rank matrices have only the zero kernel.

\hypertarget{section-37}{%
\section{41.}\label{section-37}}

If \(B\) is the zero matrix, any dimension. If \(B\) has full rank, 0.
Otherwise \(dim(\ker(B))*3\)

\hypertarget{section-38}{%
\section{42.}\label{section-38}}

\(n \times dim(\ker(B))\)

\hypertarget{section-39}{%
\section{43.}\label{section-39}}

By the description,
\(S \begin{\bmatrix}1 & 0\\0 & -1\end{\bmatrix} = \begin{bmatrix} v & -w \end{bmatrix}\).
So \(Av = v\) and \(Aw = -w\). Since \(A\) is a reflection about a line,
that means \(v\) lies entirely on the line and \(w\) is perpendicular to
it. Since \(v^{\parallel}\) and \(v^{\perp\) both have dimension 1,
\(A\) has dimension 2.

\hypertarget{section-40}{%
\section{44.}\label{section-40}}

3: 2 scalars for any basis of the plane \(A\) projects onto, a third for
the third column a scalar for the component orthogonal to the plane.

\hypertarget{section-41}{%
\section{45.}\label{section-41}}

I think

\[
\begin{bmatrix}
a & 0 & 0\\
0 & a & 0\\
0 & 0 & a
\end{bmatrix} ,
\begin{bmatrix}
0 & b & 0\\
0 & 0 & b\\
0 & 0 & 0
\end{bmatrix}
\]

\hypertarget{section-42}{%
\section{46}\label{section-42}}

Simply \((a, k)\), so dimension 2.

\hypertarget{section-43}{%
\section{47.}\label{section-43}}

Even functions satisfy the scalar property:

\[
  \begin{aligned}
    & f(-t) = f(t)\\
    &kf(-t) =kf(t)
  \end{aligned}
\]

\[
  \begin{aligned}
    & f(-t) +g(-h) = f(t) + g(h)  
  \end{aligned}
\]

They are. The scalar multiples remain part of the subspace because the
evenness condition does not apply to them. The same is true of odd
functions.

\hypertarget{section-44}{%
\section{48.}\label{section-44}}

\hypertarget{section-45}{%
\section{49.}\label{section-45}}

Yes. By linearityl, \(L(A +B) = L(A) + L(B)\), satisfying closure under
addition. And \(k(LA) = L(kA)\) So any linear transformation of a member
of \(L\) remains in \(L\), satisying the conditions for a subspace.

\hypertarget{section-46}{%
\section{50.}\label{section-46}}

\hypertarget{section-47}{%
\section{51.}\label{section-47}}

\hypertarget{section-48}{%
\section{52.}\label{section-48}}

\hypertarget{section-49}{%
\section{53.}\label{section-49}}

Say a space \(C\) of dimension \(n\) has a basis with \(n+1\) elements.
By definition, a unique linear combination of this basis describes every
member of the space. These coordinates may be mapped to vectors in
\(R^n\) using the coordinate transformation. (The standard coordinate
transformation could not be used for \(R^{n+1}\) because \(C\) has
dimension \(n\). ). Let \(V\) designate the subspace containing the
coordinate vector for every member of \(C\)'s basis. If the basis is
valid, then the members of \(V\) are linearly independent, such that
\(c_1v_1+\dots+c_{n+1}v_{n+1}=0\) has only the solution
\(c_1,\dots,c_{n+1}=0\). But a set of vectors in \(R^n\) can contain at
most \(n\) linearly independent vectors, so \(v_{n+1}\) must be
redundant. Because it is not linearly independent, \(V\) cannot form
coordinates for a basis of \(C\). But the basis would be valid if its
coordinates were \(n\) linearly independent vectors. So a linear space
of \(n\) dimensions admits at most \(n\) linearly independent elements.

\hypertarget{section-50}{%
\section{54.}\label{section-50}}

Assume \(\dim(W) = n + 1\). Then a basis for \(W\) has \(n+1\) elements.
But a basis for the containing space has only \(n\) elements, so some
members of \(W\) are not spanned by the containing space, which
contradicts its being a subspace.

\hypertarget{section-51}{%
\section{55.}\label{section-51}}

\hypertarget{section-52}{%
\section{56.}\label{section-52}}

Because the sequence is infinite, a basis with \(n\) elements cannot
acommodate a member with rank \(n+1\), so an infinite basis is needed.

\hypertarget{section-53}{%
\section{57.}\label{section-53}}

If a spanning set of \(V\) contains \(n\) elements, then a basis for
\(V\) contains \(\leq n\) elements (in the case of linear dependence).
Because a finite basis exists, \(V\) is not infinite dimensional.

\hypertarget{section-54}{%
\section{58.}\label{section-54}}

\hypertarget{section-55}{%
\section{59.}\label{section-55}}

If this is not true, then for 0 is not a neutral element for all members
of the space, which violates that axiom of linear spaces.

\hypertarget{section-56}{%
\section{60.}\label{section-56}}

\end{document}
