\PassOptionsToPackage{unicode=true}{hyperref} % options for packages loaded elsewhere
\PassOptionsToPackage{hyphens}{url}
%
\documentclass[]{article}
\usepackage{lmodern}
\usepackage{amssymb,amsmath}
\usepackage{ifxetex,ifluatex}
\usepackage{fixltx2e} % provides \textsubscript
\ifnum 0\ifxetex 1\fi\ifluatex 1\fi=0 % if pdftex
  \usepackage[T1]{fontenc}
  \usepackage[utf8]{inputenc}
  \usepackage{textcomp} % provides euro and other symbols
\else % if luatex or xelatex
  \usepackage{unicode-math}
  \defaultfontfeatures{Ligatures=TeX,Scale=MatchLowercase}
\fi
% use upquote if available, for straight quotes in verbatim environments
\IfFileExists{upquote.sty}{\usepackage{upquote}}{}
% use microtype if available
\IfFileExists{microtype.sty}{%
\usepackage[]{microtype}
\UseMicrotypeSet[protrusion]{basicmath} % disable protrusion for tt fonts
}{}
\IfFileExists{parskip.sty}{%
\usepackage{parskip}
}{% else
\setlength{\parindent}{0pt}
\setlength{\parskip}{6pt plus 2pt minus 1pt}
}
\usepackage{hyperref}
\hypersetup{
            pdftitle={Section 5.4 Problems},
            pdfauthor={Ryan Heslin},
            pdfborder={0 0 0},
            breaklinks=true}
\urlstyle{same}  % don't use monospace font for urls
\usepackage[margin=1in]{geometry}
\usepackage{color}
\usepackage{fancyvrb}
\newcommand{\VerbBar}{|}
\newcommand{\VERB}{\Verb[commandchars=\\\{\}]}
\DefineVerbatimEnvironment{Highlighting}{Verbatim}{commandchars=\\\{\}}
% Add ',fontsize=\small' for more characters per line
\usepackage{framed}
\definecolor{shadecolor}{RGB}{248,248,248}
\newenvironment{Shaded}{\begin{snugshade}}{\end{snugshade}}
\newcommand{\AlertTok}[1]{\textcolor[rgb]{0.94,0.16,0.16}{#1}}
\newcommand{\AnnotationTok}[1]{\textcolor[rgb]{0.56,0.35,0.01}{\textbf{\textit{#1}}}}
\newcommand{\AttributeTok}[1]{\textcolor[rgb]{0.77,0.63,0.00}{#1}}
\newcommand{\BaseNTok}[1]{\textcolor[rgb]{0.00,0.00,0.81}{#1}}
\newcommand{\BuiltInTok}[1]{#1}
\newcommand{\CharTok}[1]{\textcolor[rgb]{0.31,0.60,0.02}{#1}}
\newcommand{\CommentTok}[1]{\textcolor[rgb]{0.56,0.35,0.01}{\textit{#1}}}
\newcommand{\CommentVarTok}[1]{\textcolor[rgb]{0.56,0.35,0.01}{\textbf{\textit{#1}}}}
\newcommand{\ConstantTok}[1]{\textcolor[rgb]{0.00,0.00,0.00}{#1}}
\newcommand{\ControlFlowTok}[1]{\textcolor[rgb]{0.13,0.29,0.53}{\textbf{#1}}}
\newcommand{\DataTypeTok}[1]{\textcolor[rgb]{0.13,0.29,0.53}{#1}}
\newcommand{\DecValTok}[1]{\textcolor[rgb]{0.00,0.00,0.81}{#1}}
\newcommand{\DocumentationTok}[1]{\textcolor[rgb]{0.56,0.35,0.01}{\textbf{\textit{#1}}}}
\newcommand{\ErrorTok}[1]{\textcolor[rgb]{0.64,0.00,0.00}{\textbf{#1}}}
\newcommand{\ExtensionTok}[1]{#1}
\newcommand{\FloatTok}[1]{\textcolor[rgb]{0.00,0.00,0.81}{#1}}
\newcommand{\FunctionTok}[1]{\textcolor[rgb]{0.00,0.00,0.00}{#1}}
\newcommand{\ImportTok}[1]{#1}
\newcommand{\InformationTok}[1]{\textcolor[rgb]{0.56,0.35,0.01}{\textbf{\textit{#1}}}}
\newcommand{\KeywordTok}[1]{\textcolor[rgb]{0.13,0.29,0.53}{\textbf{#1}}}
\newcommand{\NormalTok}[1]{#1}
\newcommand{\OperatorTok}[1]{\textcolor[rgb]{0.81,0.36,0.00}{\textbf{#1}}}
\newcommand{\OtherTok}[1]{\textcolor[rgb]{0.56,0.35,0.01}{#1}}
\newcommand{\PreprocessorTok}[1]{\textcolor[rgb]{0.56,0.35,0.01}{\textit{#1}}}
\newcommand{\RegionMarkerTok}[1]{#1}
\newcommand{\SpecialCharTok}[1]{\textcolor[rgb]{0.00,0.00,0.00}{#1}}
\newcommand{\SpecialStringTok}[1]{\textcolor[rgb]{0.31,0.60,0.02}{#1}}
\newcommand{\StringTok}[1]{\textcolor[rgb]{0.31,0.60,0.02}{#1}}
\newcommand{\VariableTok}[1]{\textcolor[rgb]{0.00,0.00,0.00}{#1}}
\newcommand{\VerbatimStringTok}[1]{\textcolor[rgb]{0.31,0.60,0.02}{#1}}
\newcommand{\WarningTok}[1]{\textcolor[rgb]{0.56,0.35,0.01}{\textbf{\textit{#1}}}}
\usepackage{graphicx,grffile}
\makeatletter
\def\maxwidth{\ifdim\Gin@nat@width>\linewidth\linewidth\else\Gin@nat@width\fi}
\def\maxheight{\ifdim\Gin@nat@height>\textheight\textheight\else\Gin@nat@height\fi}
\makeatother
% Scale images if necessary, so that they will not overflow the page
% margins by default, and it is still possible to overwrite the defaults
% using explicit options in \includegraphics[width, height, ...]{}
\setkeys{Gin}{width=\maxwidth,height=\maxheight,keepaspectratio}
\setlength{\emergencystretch}{3em}  % prevent overfull lines
\providecommand{\tightlist}{%
  \setlength{\itemsep}{0pt}\setlength{\parskip}{0pt}}
\setcounter{secnumdepth}{0}
% Redefines (sub)paragraphs to behave more like sections
\ifx\paragraph\undefined\else
\let\oldparagraph\paragraph
\renewcommand{\paragraph}[1]{\oldparagraph{#1}\mbox{}}
\fi
\ifx\subparagraph\undefined\else
\let\oldsubparagraph\subparagraph
\renewcommand{\subparagraph}[1]{\oldsubparagraph{#1}\mbox{}}
\fi

% set default figure placement to htbp
\makeatletter
\def\fps@figure{htbp}
\makeatother


\title{Section 5.4 Problems}
\author{Ryan Heslin}
\date{2021-11-17}

\begin{document}
\maketitle

\newcommand{\abcd}{\begin{bmatrix}a&b\\
c&d\end{bmatrix}}

\newcommand{\m}[1]{\begin{bmatrix}#1\end{bmatrix}}

\newcommand{\vect}[1]{\begin{pmatrix}#1\end{pmatrix}}

\newcommand{\meq}[1]{\begin{split}#1\end{split}}

\newcommand{\bym}[1]{#1\times{m}}

\newcommand{\nby}[1]{n\times{#1}}

\newcommand{\subsp}[2]{\Bigg\{\begin{bmatrix}#1\end{bmatrix}:#2\Bigg\}}

\newcommand{\proj}[2]{\text{proj}_#1(#2)}

\newcommand{\refl}[2]{\text{refl}_#1(#2)}

\newcommand{\sumn}{\sum_{i=1}^n}

\newcommand{\dotsn}[5]{#1_{1}#3_{1}#5{#1}_{2}#3_{2}{#5}\dots{#5}#1_{#2}#3_{#4}}

\hypertarget{section}{%
\subsection{1.}\label{section}}

\begin{\bmatrix}
1\\
-2
\end{\bmatrix}

\hypertarget{section-1}{%
\subsection{3.}\label{section-1}}

Say we have bases of \(V\) and \(V^{\perp}\). Then the union of the
bases forms a basis for \(R^n\), because all vectors in \(R^n\) are
either linear combinations of \(V\) or are orthogonal to, except for 0,
which is both.

\hypertarget{section-2}{%
\subsection{4.}\label{section-2}}

Yes. The image of \(A^T\) is all vectors in \(R^m\) obtained by
\(A^Tx\). \(\ker(A)^{\perp}\) is all vectors in \(R^m\) for which
\(Ax \neq 0\).

\hypertarget{section-3}{%
\subsection{5.}\label{section-3}}

The solution space \(V\) of

\[
  \begin{aligned}
    & x_1 + x_2 +x_3 +x_4 = 0\\
    & x_1 +2x_2 +5x_3 +4x_4 =0
  \end{aligned}
\] is the kernel of the matrix. So \(V^{\perp} = \text{im}A^T\):

\[ \begin{bmatrix}
  1 & 1\\
  1 & 2\\
  1 & 5\\
  1 & 4\\
\end{bmatrix}\]

\hypertarget{section-4}{%
\subsection{6.}\label{section-4}}

If \(A\) is \(n \times m\), then \(im(A) =(AA^T\) is true. \$
\text{im}A\^{}T\$ is the orthogonal complement of the kernel, so it
contains all nonzero \(x\) for which \(Ax \neq 0\) - in other words, all
vectors in \((A^ {\perp})^{\perp}\)

\hypertarget{section-5}{%
\subsection{7.}\label{section-5}}

If a matrix is symmetric, the image and kernel are orthogonal
complements, because \(A^T=A\), so \(\ker(A^T) = \ker(A)\). Likewise,
the row space and \(\ker(A)\) are orthogonal complements.

\hypertarget{section-6}{%
\subsection{8.}\label{section-6}}

\begin{enumerate}
\def\labelenumi{\alph{enumi}.}
\item
  \(A^+ = (A^TA)^{-1}A^T\)
\item
  \[
    \begin{aligned}
   & A^+ = (A^TA)^{-1}A^T\\
   & = A^{-1}(A^T)^{-1}A^T\\
   & = A^{-1}
    \end{aligned}
  \]
\item
  \((A^TA)^{-1}A^TAx = x\)
\item
  \(A(A^TA)^{-1}A^Ty= y^{\parallel}\)
\item
  \[L^+ = \begin{bmatrix}
    1 & 0 & 0\\
    0 & 1 & 0
  \end{bmatrix}\]
\end{enumerate}

\hypertarget{section-7}{%
\subsection{9.}\label{section-7}}

\hypertarget{b.}{%
\subsubsection{b.}\label{b.}}

They are orthogonal complements.

\hypertarget{c.}{%
\subsubsection{c.}\label{c.}}

It is a absis for \(im(A^T)\).

\hypertarget{d.}{%
\subsubsection{d.}\label{d.}}

\[
\begin{\bmatrix}
1\\
2
\end{\bmatrix}
\]

\hypertarget{e.}{%
\subsubsection{e.}\label{e.}}

Being the minimal solution, it is the shortest.

\hypertarget{section-8}{%
\subsection{10.}\label{section-8}}

\begin{enumerate}
\def\labelenumi{\alph{enumi}.}
\item
  If \(x_0\) is in \(\ker A^{\perp}\), then it lies in the image of the
  transpose, the orthogonal complement. Then since \(x = x_h + x_0\), if
  we set \(x_h=0\) (the portion in the kernel), then \(x_0\) lies
  entirely in the image of the transpose. all vectors for which that
  lead to nonzero \(b\), as well as \(0_m\)
\item
  \(0_m\) is the only vector shared among \((\ker(A^{\perp}))\) and
  \(\ker(A)\), so \(x_0\) lies entirely in the image of the transpose
  only if \(x_h=0\).
\item
  For all linear combinations \(x_0\) of \(\ker(A)\), \(Ax_0=0\) by
  definition. So combinations of \(\ker(A)\) can be freely added to
  \(x_h\) without impacting the solution, since
  \(A(x_0 +x_h)=Ax_0 + Ax_h=0+Ax_h\). Since all nonzero vectors have
  nonzero length, this makes \(Ax_h\) the shortest solution.
\end{enumerate}

\hypertarget{section-9}{%
\subsection{11.}\label{section-9}}

\begin{enumerate}
\def\labelenumi{\alph{enumi}.}
\tightlist
\item
  Given the definition of the minimal solution, the minimal
  least-squares solution is the next best thing: the one solution to
  \((A^TA)^{-1}A^Tx\) lying in \(\ker(A)^{\perp})\). That is the image
  of the transpose, so this unique solution is purely a linear
  combination of \(A^T\), without any vectors from \(\ker(A)\).
\end{enumerate}

Had these backwards initially.

\begin{enumerate}
\def\labelenumi{\alph{enumi}.}
\setcounter{enumi}{1}
\item
  \((A^TA)^{-1}A^TA=I\)
\item
  \(A(A^TA)^{-1}A^T\)
\item
  The image is \(R^n\), the kernel is \(\ker(A)\)
\end{enumerate}

e.The first two elements of \(y\).

\hypertarget{section-10}{%
\subsection{12.}\label{section-10}}

The minimal least-squares solution of a system is the shortest solution
\(x^+\) that yields an \(Ax^+\) the shortest distance from \(b\). It
always lies in \((\ker A)^{\perp}\) because it is the one and only
\(x^+\) that lies entirely in \(A\)'s row space.

\hypertarget{section-11}{%
\section{13.}\label{section-11}}

\begin{enumerate}
\def\labelenumi{\alph{enumi}.}
\tightlist
\item
  \(L(x) = y\) is linear, so \(L(y_1 + y_2)\) is the minimal
  least-squares solution of \(L(x) = y_1 +y_2\), which is the sum of the
  separate least-squares solutions for \(y_1\) and \(y_2\). For the
  second property:
\end{enumerate}

\[
  \begin{aligned}
    & L(kx) = kL(x)\\
    & L^+(L(kx)) = L^+(k(L(x))) = kL^+(L(x))
  \end{aligned}
\]

(I was initially not quite right on these before looking up the
answers). b. \(L^+(L(X))\) is the minimum least-squares solution of
\(L(x) = L(X)\) - that is, \(x\). More correctly, the projection of
\(x\) onto the image of \(A^T\).

\begin{enumerate}
\def\labelenumi{\alph{enumi}.}
\setcounter{enumi}{2}
\item
  The projection of \(y\) onto the image of the row space.
\item
  The image and kernel of \(L^+\) are the same as those of \(A^T\).
\item
  If \[ L(x) = \begin{bmatrix}
  2 & 0 & 0\\
  0 & 0 & 0
  \end{bmatrix}x\], then the pseudoinverse is just
  \(\begin{bmatrix}1/2 & 0\\0&0\\0&0\end{bmatrix}\)
\end{enumerate}

\hypertarget{section-12}{%
\subsection{15.}\label{section-12}}

It is the pseudoinverse, \((A^TA)^{-1}A^T\). We have:

\[
  \begin{aligned}
    & (A^TA)^{-1}A^TA =I
  \end{aligned}
\]

\hypertarget{section-13}{%
\subsection{17.}\label{section-13}}

Yes. If this were not true, then \(\dim(\ker(A^T)) \geq \dim(\ker(A))\).
This is impossible, because the orthogonal complement of \(A\)'s image
is \(\ker(A^T)\) So no \(x\) solving \(A^Tx=0\) may be produced by a
linear combination of \(A\). Therefore, \(\ker(A^TA) = \ker(A)\), and
both matrices have \(m\) columns, so ranks are equal as well.

\hypertarget{section-14}{%
\subsection{18.}\label{section-14}}

Yes. We proved above \(A^TA\) and \(A\) have equal rank. \(\ker(A^T)\)
is the orthogonal complement of the image of \(A^T\), so any \(x\) for
which \(Ax=0\) cannot come from a linear combination of \(A^T\) (except
for \(0_m\)). So the kernel does not expand, and rank remains the same.

\hypertarget{section-15}{%
\subsection{19.}\label{section-15}}

\[
        \begin{\bmatrix}
        1\\
        1
        \end{\bmatrix}
\]

\hypertarget{section-16}{%
\subsection{28.}\label{section-16}}

For an orthonormal basis, the least squares solution is \(b\), since
\(A(A^TA)^{-1}A^T=I\).

\hypertarget{section-17}{%
\subsection{34.}\label{section-17}}

\begin{Shaded}
\begin{Highlighting}[]
\NormalTok{x <-}\StringTok{ }\NormalTok{b <-}\StringTok{ }\KeywordTok{c}\NormalTok{(}\DecValTok{0}\NormalTok{, }\FloatTok{0.5}\NormalTok{, }\DecValTok{1}\NormalTok{, }\FloatTok{1.5}\NormalTok{, }\DecValTok{2}\NormalTok{, }\FloatTok{2.5}\NormalTok{, }\DecValTok{3}\NormalTok{)}
\NormalTok{X <-}\StringTok{ }\KeywordTok{cbind}\NormalTok{(}\DecValTok{1}\NormalTok{, }\KeywordTok{sin}\NormalTok{(x), }\KeywordTok{cos}\NormalTok{(x), }\KeywordTok{sin}\NormalTok{(}\DecValTok{2} \OperatorTok{*}\StringTok{ }\NormalTok{x), }\KeywordTok{cos}\NormalTok{(}\DecValTok{2} \OperatorTok{*}\StringTok{ }\NormalTok{x))}

\KeywordTok{solve}\NormalTok{(}\KeywordTok{t}\NormalTok{(X) }\OperatorTok\StringTok{ }\NormalTok{X) }\OperatorTok\StringTok{ }\KeywordTok{t}\NormalTok{(X) }\OperatorTok\StringTok{ }\NormalTok{b}
\end{Highlighting}
\end{Shaded}

\begin{verbatim}
            [,1]
[1,]  1.50000000
[2,]  0.10897422
[3,] -1.53669122
[4,]  0.30269197
[5,]  0.04314769
\end{verbatim}

\hypertarget{section-18}{%
\subsection{36.}\label{section-18}}

I fit a model predicting day length by time of year. The error vector is
surprisingly small.

\begin{Shaded}
\begin{Highlighting}[]
\NormalTok{fit <-}\StringTok{ }\ControlFlowTok{function}\NormalTok{(A) \{}
    \KeywordTok{solve}\NormalTok{(}\KeywordTok{t}\NormalTok{(A) }\OperatorTok\StringTok{ }\NormalTok{A) }\OperatorTok\StringTok{ }\KeywordTok{t}\NormalTok{(A)}
\NormalTok{\}}
\NormalTok{days <-}\StringTok{ }\KeywordTok{c}\NormalTok{(}\DecValTok{32}\NormalTok{, }\DecValTok{77}\NormalTok{, }\DecValTok{121}\NormalTok{, }\DecValTok{152}\NormalTok{)}
\NormalTok{b <-}\StringTok{ }\KeywordTok{c}\NormalTok{(}\DecValTok{10}\NormalTok{, }\DecValTok{12}\NormalTok{, }\DecValTok{14}\NormalTok{, }\DecValTok{15}\NormalTok{)}

\NormalTok{A <-}\StringTok{ }\KeywordTok{cbind}\NormalTok{(}\KeywordTok{rep}\NormalTok{(}\DecValTok{1}\NormalTok{, }\DecValTok{4}\NormalTok{), }\KeywordTok{sin}\NormalTok{(((}\DecValTok{2} \OperatorTok{*}\StringTok{ }\NormalTok{pi)}\OperatorTok{/}\DecValTok{366}\NormalTok{) }\OperatorTok{*}\StringTok{ }\NormalTok{days), }\KeywordTok{cos}\NormalTok{(((}\DecValTok{2} \OperatorTok{*}
\StringTok{    }\NormalTok{pi)}\OperatorTok{/}\DecValTok{366}\NormalTok{) }\OperatorTok{*}\StringTok{ }\NormalTok{days))}

\NormalTok{betas <-}\StringTok{ }\KeywordTok{fit}\NormalTok{(A) }\OperatorTok\StringTok{ }\NormalTok{b}

\NormalTok{b }\OperatorTok{-}\StringTok{ }\NormalTok{A }\OperatorTok\StringTok{ }\NormalTok{betas}
\end{Highlighting}
\end{Shaded}

\begin{verbatim}
        [,1]
\end{verbatim}

{[}1,{]} -0.01328337 {[}2,{]} 0.03559242 {[}3,{]} -0.04424210 {[}4,{]}
0.02193305

\hypertarget{section-19}{%
\subsection{39.}\label{section-19}}

Another exponential fit problem.

\begin{enumerate}
\def\labelenumi{\alph{enumi}.}
\item
\end{enumerate}

\begin{Shaded}
\begin{Highlighting}[]
\NormalTok{A <-}\StringTok{ }\KeywordTok{cbind}\NormalTok{(}\KeywordTok{rep}\NormalTok{(}\DecValTok{1}\NormalTok{, }\DecValTok{5}\NormalTok{), }\KeywordTok{log}\NormalTok{(}\KeywordTok{c}\NormalTok{(}\FloatTok{6e+05}\NormalTok{, }\FloatTok{2e+05}\NormalTok{, }\DecValTok{60000}\NormalTok{, }\DecValTok{10000}\NormalTok{,}
    \DecValTok{2500}\NormalTok{)))}
\NormalTok{z <-}\StringTok{ }\KeywordTok{c}\NormalTok{(}\DecValTok{5}\NormalTok{, }\DecValTok{12}\NormalTok{, }\DecValTok{25}\NormalTok{, }\DecValTok{60}\NormalTok{, }\DecValTok{250}\NormalTok{)}

\NormalTok{betas <-}\StringTok{ }\KeywordTok{fit}\NormalTok{(A) }\OperatorTok\StringTok{ }\KeywordTok{log}\NormalTok{(z)}
\end{Highlighting}
\end{Shaded}

\begin{enumerate}
\def\labelenumi{\alph{enumi}.}
\setcounter{enumi}{2}
\item
\end{enumerate}

The exponential base of the fitted function is about 0.5, very close to
the theoretical \(a = k \sqrt g\).

\begin{Shaded}
\begin{Highlighting}[]
\NormalTok{k <-}\StringTok{ }\KeywordTok{exp}\NormalTok{(betas[}\DecValTok{1}\NormalTok{])}
\NormalTok{g <-}\StringTok{ }\KeywordTok{exp}\NormalTok{(betas[}\DecValTok{2}\NormalTok{])}
\KeywordTok{sprintf}\NormalTok{(}\StringTok{"k = %.2f, g = %.2f"}\NormalTok{, k, g)}
\end{Highlighting}
\end{Shaded}

\begin{verbatim}
[1] "k = 40263.28, g = 0.51"
\end{verbatim}

\hypertarget{section-20}{%
\subsection{41.}\label{section-20}}

Let's predict the US national debt!

\begin{Shaded}
\begin{Highlighting}[]
\NormalTok{A <-}\StringTok{ }\KeywordTok{cbind}\NormalTok{(}\KeywordTok{rep}\NormalTok{(}\DecValTok{1}\NormalTok{, }\DecValTok{4}\NormalTok{), }\KeywordTok{seq}\NormalTok{(}\DecValTok{0}\NormalTok{, }\DecValTok{30}\NormalTok{, }\DataTypeTok{by =} \DecValTok{10}\NormalTok{))}

\NormalTok{b <-}\StringTok{ }\KeywordTok{log}\NormalTok{(}\KeywordTok{c}\NormalTok{(}\DecValTok{533}\NormalTok{, }\DecValTok{1823}\NormalTok{, }\DecValTok{4974}\NormalTok{, }\DecValTok{7933}\NormalTok{))}
\NormalTok{betas <-}\StringTok{ }\KeywordTok{fit}\NormalTok{(A) }\OperatorTok\StringTok{ }\NormalTok{b}

\KeywordTok{cbind}\NormalTok{(A[, }\DecValTok{2}\NormalTok{], }\KeywordTok{exp}\NormalTok{(b)) }\OperatorTok\StringTok{ }\NormalTok{betas}
\end{Highlighting}
\end{Shaded}

\begin{verbatim}
      [,1]
\end{verbatim}

{[}1,{]} 48.52718 {[}2,{]} 230.51271 {[}3,{]} 581.93366 {[}4,{]}
915.87389

\begin{Shaded}
\begin{Highlighting}[]
\CommentTok{# exp(betas[1] +betas[2] * c(A[,2], 40))}
\end{Highlighting}
\end{Shaded}

For 2015, the model predicts a debt of more than \$24 trillion.

\hypertarget{section-21}{%
\subsection{42.}\label{section-21}}

Since \((A^T)^T=A\): \[
  \begin{aligned}
    & L(x) = Ax\\
    & L(A^Tx) = Ax\\
    & L^{-1}(Ax) = A^Tx
  \end{aligned}
\]

\end{document}
