\PassOptionsToPackage{unicode=true}{hyperref} % options for packages loaded elsewhere
\PassOptionsToPackage{hyphens}{url}
%
\documentclass[]{article}
\usepackage{lmodern}
\usepackage{amssymb,amsmath}
\usepackage{ifxetex,ifluatex}
\usepackage{fixltx2e} % provides \textsubscript
\ifnum 0\ifxetex 1\fi\ifluatex 1\fi=0 % if pdftex
  \usepackage[T1]{fontenc}
  \usepackage[utf8]{inputenc}
  \usepackage{textcomp} % provides euro and other symbols
\else % if luatex or xelatex
  \usepackage{unicode-math}
  \defaultfontfeatures{Ligatures=TeX,Scale=MatchLowercase}
\fi
% use upquote if available, for straight quotes in verbatim environments
\IfFileExists{upquote.sty}{\usepackage{upquote}}{}
% use microtype if available
\IfFileExists{microtype.sty}{%
\usepackage[]{microtype}
\UseMicrotypeSet[protrusion]{basicmath} % disable protrusion for tt fonts
}{}
\IfFileExists{parskip.sty}{%
\usepackage{parskip}
}{% else
\setlength{\parindent}{0pt}
\setlength{\parskip}{6pt plus 2pt minus 1pt}
}
\usepackage{hyperref}
\hypersetup{
            pdftitle={Section 2.5 Problems},
            pdfauthor={Ryan Heslin},
            pdfborder={0 0 0},
            breaklinks=true}
\urlstyle{same}  % don't use monospace font for urls
\usepackage[margin=1in]{geometry}
\usepackage{color}
\usepackage{fancyvrb}
\newcommand{\VerbBar}{|}
\newcommand{\VERB}{\Verb[commandchars=\\\{\}]}
\DefineVerbatimEnvironment{Highlighting}{Verbatim}{commandchars=\\\{\}}
% Add ',fontsize=\small' for more characters per line
\usepackage{framed}
\definecolor{shadecolor}{RGB}{255,255,255}
\newenvironment{Shaded}{\begin{snugshade}}{\end{snugshade}}
\newcommand{\AlertTok}[1]{\textcolor[rgb]{0.75,0.01,0.01}{\textbf{\colorbox[rgb]{0.97,0.90,0.90}{#1}}}}
\newcommand{\AnnotationTok}[1]{\textcolor[rgb]{0.79,0.38,0.79}{#1}}
\newcommand{\AttributeTok}[1]{\textcolor[rgb]{0.00,0.34,0.68}{#1}}
\newcommand{\BaseNTok}[1]{\textcolor[rgb]{0.69,0.50,0.00}{#1}}
\newcommand{\BuiltInTok}[1]{\textcolor[rgb]{0.39,0.29,0.61}{\textbf{#1}}}
\newcommand{\CharTok}[1]{\textcolor[rgb]{0.57,0.30,0.62}{#1}}
\newcommand{\CommentTok}[1]{\textcolor[rgb]{0.54,0.53,0.53}{#1}}
\newcommand{\CommentVarTok}[1]{\textcolor[rgb]{0.00,0.58,1.00}{#1}}
\newcommand{\ConstantTok}[1]{\textcolor[rgb]{0.67,0.33,0.00}{#1}}
\newcommand{\ControlFlowTok}[1]{\textcolor[rgb]{0.12,0.11,0.11}{\textbf{#1}}}
\newcommand{\DataTypeTok}[1]{\textcolor[rgb]{0.00,0.34,0.68}{#1}}
\newcommand{\DecValTok}[1]{\textcolor[rgb]{0.69,0.50,0.00}{#1}}
\newcommand{\DocumentationTok}[1]{\textcolor[rgb]{0.38,0.47,0.50}{#1}}
\newcommand{\ErrorTok}[1]{\textcolor[rgb]{0.75,0.01,0.01}{\underline{#1}}}
\newcommand{\ExtensionTok}[1]{\textcolor[rgb]{0.00,0.58,1.00}{\textbf{#1}}}
\newcommand{\FloatTok}[1]{\textcolor[rgb]{0.69,0.50,0.00}{#1}}
\newcommand{\FunctionTok}[1]{\textcolor[rgb]{0.39,0.29,0.61}{#1}}
\newcommand{\ImportTok}[1]{\textcolor[rgb]{1.00,0.33,0.00}{#1}}
\newcommand{\InformationTok}[1]{\textcolor[rgb]{0.69,0.50,0.00}{#1}}
\newcommand{\KeywordTok}[1]{\textcolor[rgb]{0.12,0.11,0.11}{\textbf{#1}}}
\newcommand{\NormalTok}[1]{\textcolor[rgb]{0.12,0.11,0.11}{#1}}
\newcommand{\OperatorTok}[1]{\textcolor[rgb]{0.12,0.11,0.11}{#1}}
\newcommand{\OtherTok}[1]{\textcolor[rgb]{0.00,0.43,0.16}{#1}}
\newcommand{\PreprocessorTok}[1]{\textcolor[rgb]{0.00,0.43,0.16}{#1}}
\newcommand{\RegionMarkerTok}[1]{\textcolor[rgb]{0.00,0.34,0.68}{\colorbox[rgb]{0.88,0.91,0.97}{#1}}}
\newcommand{\SpecialCharTok}[1]{\textcolor[rgb]{0.24,0.68,0.91}{#1}}
\newcommand{\SpecialStringTok}[1]{\textcolor[rgb]{1.00,0.33,0.00}{#1}}
\newcommand{\StringTok}[1]{\textcolor[rgb]{0.75,0.01,0.01}{#1}}
\newcommand{\VariableTok}[1]{\textcolor[rgb]{0.00,0.34,0.68}{#1}}
\newcommand{\VerbatimStringTok}[1]{\textcolor[rgb]{0.75,0.01,0.01}{#1}}
\newcommand{\WarningTok}[1]{\textcolor[rgb]{0.75,0.01,0.01}{#1}}
\usepackage{graphicx,grffile}
\makeatletter
\def\maxwidth{\ifdim\Gin@nat@width>\linewidth\linewidth\else\Gin@nat@width\fi}
\def\maxheight{\ifdim\Gin@nat@height>\textheight\textheight\else\Gin@nat@height\fi}
\makeatother
% Scale images if necessary, so that they will not overflow the page
% margins by default, and it is still possible to overwrite the defaults
% using explicit options in \includegraphics[width, height, ...]{}
\setkeys{Gin}{width=\maxwidth,height=\maxheight,keepaspectratio}
\setlength{\emergencystretch}{3em}  % prevent overfull lines
\providecommand{\tightlist}{%
  \setlength{\itemsep}{0pt}\setlength{\parskip}{0pt}}
\setcounter{secnumdepth}{0}
% Redefines (sub)paragraphs to behave more like sections
\ifx\paragraph\undefined\else
\let\oldparagraph\paragraph
\renewcommand{\paragraph}[1]{\oldparagraph{#1}\mbox{}}
\fi
\ifx\subparagraph\undefined\else
\let\oldsubparagraph\subparagraph
\renewcommand{\subparagraph}[1]{\oldsubparagraph{#1}\mbox{}}
\fi

% set default figure placement to htbp
\makeatletter
\def\fps@figure{htbp}
\makeatother

\newcommand{\abcd}{\begin{bmatrix}a&b\\
c&d\end{bmatrix}}

\newcommand{\m}[1]{\begin{bmatrix}#1\end{bmatrix}}

\newcommand{\vect}[1]{\begin{pmatrix}#1\end{pmatrix}}

\newcommand{\meq}[1]{\begin{split}#1\end{split}}

\title{Section 2.5 Problems}
\author{Ryan Heslin}
\date{November 17, 2021}

\begin{document}
\maketitle

An incidence matrix has a row for every edge. For each row, there is -1
in the column indexed by the ``from'' node, and 1 in the column
corresponding to the ``to'' node.

The dimensions are edges \(\times\) nodes.

The vector \(\mathbold { 1 }\) is always in the kernel because an
arbitrary constant can always be added to the solution. The graph only
expresses relative relationships, not absolute ones. So rank is at most
\(m - 1\).

A solution to \(Ax = 0\) represents a distirbution for which it is
impossible to find the actual potentials. A solution to \(A^Ty = 0\)
represents a distribution that cirulates endlessly (a cycle). A basis
for the row space thus contains those edges for which cycles are
impossible.

For an incidence, the dimensions are:

\begin{tabular}{c|c}
Subspace & Dimension \\
Kernel & $1; \mathbold { 1 }$ \\ 
Row & $m-1$\\
Image & $m-1$\\
Transpose kernel & $n - m + 1$
\end{tabular}

Euler's formula states that in a directed graph, the numbers of edges,
nodes, and loops sum to 1, apparent from the dimensions above.

A network is a graph where each edge is assigned a weight by the
diagonal \(n \times n\) matrix \(C\).

Kirchoff's law states that the inflows to any node 9the potentials) sum
to 0. From Ohm's law, there are implicit equations for equilibrium:

\[
    \begin{aligned}
    & y = C(b - Ax) \text{or} C^{-1}y + Ax = b\\ 
    & A^Ty = f
    \end{aligned}
\]

The first says any edge dsitribtuion is obtained by scaling the kernel
component of a potentials vector by \(C\). The second says any
potentials vector can be recoevered. By some rearrangement:

\[
    A^TCAx = A^TCb - f
\] This holds only if the last element of the potentials vector \(x _m\)
= 0, ignoring the last column of \(A\). The reduced graph has rank
\(m\), so it is a tree - no loops are possible.

The least squares problem \(Ax -b\) is a special case of this.

\hypertarget{section}{%
\subsection{1.}\label{section}}

The graph is \[
    \begin{bmatrix}
    1 & -1 & 0\\
    0 & 1 & -1\\
    1 & 0 & -1
    \end{bmatrix}
\] and its kernel is \(\mathbold 1\). The kernel of \(A^T\) is
\(\begin{bmatrix}-1\\-1\\1\end{bmatrix}\)

\hypertarget{section-1}{%
\subsection{2.}\label{section-1}}

From the rows and columns, it is obvious summing the first two gives the
third.

\hypertarget{section-2}{%
\subsection{3.}\label{section-2}}

Again, immediate from attempting to sum the rows. It means the last
difference has to cancel the previous ones.

\hypertarget{section-3}{%
\subsection{4.}\label{section-3}}

\[
\begin{bmatrix}
2 & -1 & -1\\
-1 & 2 & -1\\
-1 & -1 & 2
\end{bmatrix}
\] A few elimination steps gives a zero row. The submatrix has inverse

\[
    \frac {1}{5}\begin{bmatrix}
    2 & 1\\
    1 & 2
    \end{bmatrix}
\]

\hypertarget{section-4}{%
\subsection{5.}\label{section-4}}

Now it's

\[
    \frac {1}{5(c_1 + c_2)}\begin{bmatrix}
    2c_1 & 1c_2\\
    1c_1 & 2c_2
    \end{bmatrix}
\]

\hypertarget{section-5}{%
\subsection{6.}\label{section-5}}

\begin{Shaded}
\begin{Highlighting}[]
\NormalTok{A <-}\StringTok{ }\KeywordTok{matrix}\NormalTok{(}\KeywordTok{c}\NormalTok{(}
  \DecValTok{-1}\NormalTok{, }\DecValTok{1}\NormalTok{, }\DecValTok{0}\NormalTok{, }\DecValTok{0}\NormalTok{,}
  \DecValTok{-1}\NormalTok{, }\DecValTok{0}\NormalTok{, }\DecValTok{1}\NormalTok{, }\DecValTok{0}\NormalTok{,}
  \DecValTok{0}\NormalTok{, }\DecValTok{-1}\NormalTok{, }\DecValTok{1}\NormalTok{, }\DecValTok{0}\NormalTok{,}
  \DecValTok{0}\NormalTok{, }\DecValTok{-1}\NormalTok{, }\DecValTok{0}\NormalTok{, }\DecValTok{1}\NormalTok{,}
  \DecValTok{-1}\NormalTok{, }\DecValTok{0}\NormalTok{, }\DecValTok{0}\NormalTok{, }\DecValTok{1}\NormalTok{,}
  \DecValTok{0}\NormalTok{, }\DecValTok{0}\NormalTok{, }\DecValTok{-1}\NormalTok{, }\DecValTok{1}
\NormalTok{), }\DataTypeTok{nrow =} \DecValTok{6}\NormalTok{, }\DataTypeTok{byrow =} \OtherTok{TRUE}\NormalTok{)}
\end{Highlighting}
\end{Shaded}

\hypertarget{section-6}{%
\subsection{7.}\label{section-6}}

Probably screwed up, but the kernel is \[
 \begin{bmatrix}
 -1 & 1 & 1\\
 -1 & 0 & 1\\
 1 & 0 & 0\\
 0 & -1 & -1\\
 0 & 1 & 0\\
 0 & 0 & 1
 \end{bmatrix}
\]

\hypertarget{section-7}{%
\subsection{8.}\label{section-7}}

Row and column are both \(m-1 = 3\). Kernel is 1, making transpose
kernel \(6 - (4-1) = 3\).

\hypertarget{section-8}{%
\subsection{9.}\label{section-8}}

\begin{Shaded}
\begin{Highlighting}[]
\NormalTok{C <-}\StringTok{ }\KeywordTok{diag}\NormalTok{(}\DataTypeTok{x =} \KeywordTok{complex}\NormalTok{(}\DecValTok{6}\NormalTok{, }\DecValTok{0}\NormalTok{, }\DecValTok{1}\OperatorTok{:}\DecValTok{6}\NormalTok{))}
\KeywordTok{t}\NormalTok{(A) }\OperatorTok\StringTok{ }\NormalTok{C }\OperatorTok\StringTok{ }\NormalTok{A}
\end{Highlighting}
\end{Shaded}

\begin{verbatim}
     [,1] [,2]  [,3]  [,4]
[1,] 0+8i 0-1i 0- 2i 0- 5i
[2,] 0-1i 0+8i 0- 3i 0- 4i
[3,] 0-2i 0-3i 0+11i 0- 6i
[4,] 0-5i 0-4i 0- 6i 0+15i
\end{verbatim}

I believe it's the sum of the constat associated with all nodes that
eventually reach that node.

\hypertarget{section-9}{%
\subsection{10.}\label{section-9}}

REmove one row and the matrix has full rank, so the remaining rows are a
basis for the potentials.

\hypertarget{section-10}{%
\subsection{11.}\label{section-10}}

The big ugly system:

\[
    \begin{aligned}
    & y_1 = -x_1 + x_2 \\ 
    & 1/2y_2 = -x_1 + x_3 \\ 
    & 1/2 y_3 = x_2 \\ 
    & y_4 = -x_3 \\ 
    & -y_1 - y_2 = f_1 \\ 
    & y_1 + y_3 = f_2 \\ 
    & y_2 - y_4 = f_4
    \end{aligned}
\]

I tried by hand but decided it wasn't worth it.

\begin{Shaded}
\begin{Highlighting}[]
\NormalTok{A <-}\StringTok{ }\KeywordTok{matrix}\NormalTok{(}\KeywordTok{c}\NormalTok{(}\OperatorTok{-}\DecValTok{1}\NormalTok{, }\DecValTok{1}\NormalTok{, }\DecValTok{0}\NormalTok{, }\DecValTok{-1}\NormalTok{, }\DecValTok{0}\NormalTok{, }\DecValTok{1}\NormalTok{, }\DecValTok{0}\NormalTok{, }\DecValTok{1}\NormalTok{, }\DecValTok{0}\NormalTok{, }\DecValTok{0}\NormalTok{, }\DecValTok{0}\NormalTok{, }\DecValTok{-1}\NormalTok{), }\DataTypeTok{nrow =} \DecValTok{4}\NormalTok{, }\DataTypeTok{byrow =} \OtherTok{TRUE}\NormalTok{)}
\NormalTok{C <-}\StringTok{ }\KeywordTok{diag}\NormalTok{(}\DataTypeTok{x =} \KeywordTok{c}\NormalTok{(}\DecValTok{1}\NormalTok{, }\DecValTok{2}\NormalTok{, }\DecValTok{2}\NormalTok{, }\DecValTok{1}\NormalTok{))}
\NormalTok{f <-}\StringTok{ }\KeywordTok{c}\NormalTok{(}\DecValTok{1}\NormalTok{, }\DecValTok{1}\NormalTok{, }\DecValTok{6}\NormalTok{)}
\NormalTok{solution <-}\StringTok{ }\KeywordTok{solve}\NormalTok{(}\KeywordTok{t}\NormalTok{(A) }\OperatorTok\StringTok{ }\NormalTok{C }\OperatorTok\StringTok{ }\NormalTok{A, }\OperatorTok{-}\NormalTok{f)}
\NormalTok{solution}
\end{Highlighting}
\end{Shaded}

\begin{verbatim}
[1] -4.000000 -1.666667 -4.666667
\end{verbatim}

\begin{Shaded}
\begin{Highlighting}[]
\NormalTok{A }\OperatorTok\StringTok{ }\NormalTok{solution}
\end{Highlighting}
\end{Shaded}

\begin{verbatim}
           [,1]
[1,]  2.3333333
[2,] -0.6666667
[3,] -1.6666667
[4,]  4.6666667
\end{verbatim}

Converting back to \(f\):

\begin{Shaded}
\begin{Highlighting}[]
\NormalTok{y <-}\StringTok{ }\NormalTok{C }\OperatorTok\StringTok{ }\NormalTok{A }\OperatorTok\StringTok{ }\NormalTok{solution}
\KeywordTok{t}\NormalTok{(A) }\OperatorTok\StringTok{ }\NormalTok{y}
\end{Highlighting}
\end{Shaded}

\begin{verbatim}
     [,1]
[1,]   -1
[2,]   -1
[3,]   -6
\end{verbatim}

\hypertarget{section-11}{%
\subsection{12.}\label{section-11}}

Rank is 6. \(\ker(A)\) has dimension 1, \(\ker(A^T)\) dimension 6. We'd
need to lop off \(12-6 = 6\) edges.

\hypertarget{section-12}{%
\subsection{13.}\label{section-12}}

\hypertarget{section-13}{%
\subsection{14.}\label{section-13}}

\hypertarget{section-14}{%
\subsection{15.}\label{section-14}}

\hypertarget{section-15}{%
\subsection{16.}\label{section-15}}

\(n \choose 2\)

\hypertarget{section-16}{%
\subsection{17.}\label{section-16}}

\hypertarget{section-17}{%
\subsection{18.}\label{section-17}}

Other nodes, connected.

\hypertarget{section-18}{%
\subsection{19.}\label{section-18}}

A constant can be added to any initial potential without distrubing the
solution. The kernel has dimension \(n-1\).

\hypertarget{section-19}{%
\subsection{20.}\label{section-19}}

That is \(6 \choose 2\).

\hypertarget{section-20}{%
\subsection{21.}\label{section-20}}

I won't write the matrix, but \(A_i \cdot A_j\) adds 1 for every
overlapping pair of nodes, and subsequent multiplications compound.

\end{document}
