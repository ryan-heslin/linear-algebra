\PassOptionsToPackage{unicode=true}{hyperref} % options for packages loaded elsewhere
\PassOptionsToPackage{hyphens}{url}
%
\documentclass[]{article}
\usepackage{lmodern}
\usepackage{amssymb,amsmath}
\usepackage{ifxetex,ifluatex}
\usepackage{fixltx2e} % provides \textsubscript
\ifnum 0\ifxetex 1\fi\ifluatex 1\fi=0 % if pdftex
  \usepackage[T1]{fontenc}
  \usepackage[utf8]{inputenc}
  \usepackage{textcomp} % provides euro and other symbols
\else % if luatex or xelatex
  \usepackage{unicode-math}
  \defaultfontfeatures{Ligatures=TeX,Scale=MatchLowercase}
\fi
% use upquote if available, for straight quotes in verbatim environments
\IfFileExists{upquote.sty}{\usepackage{upquote}}{}
% use microtype if available
\IfFileExists{microtype.sty}{%
\usepackage[]{microtype}
\UseMicrotypeSet[protrusion]{basicmath} % disable protrusion for tt fonts
}{}
\IfFileExists{parskip.sty}{%
\usepackage{parskip}
}{% else
\setlength{\parindent}{0pt}
\setlength{\parskip}{6pt plus 2pt minus 1pt}
}
\usepackage{hyperref}
\hypersetup{
            pdftitle={Linear Algebra Chapter 1 Problems},
            pdfauthor={Ryan Heslin},
            pdfborder={0 0 0},
            breaklinks=true}
\urlstyle{same}  % don't use monospace font for urls
\usepackage[margin=1in]{geometry}
\usepackage{color}
\usepackage{fancyvrb}
\newcommand{\VerbBar}{|}
\newcommand{\VERB}{\Verb[commandchars=\\\{\}]}
\DefineVerbatimEnvironment{Highlighting}{Verbatim}{commandchars=\\\{\}}
% Add ',fontsize=\small' for more characters per line
\usepackage{framed}
\definecolor{shadecolor}{RGB}{248,248,248}
\newenvironment{Shaded}{\begin{snugshade}}{\end{snugshade}}
\newcommand{\AlertTok}[1]{\textcolor[rgb]{0.94,0.16,0.16}{#1}}
\newcommand{\AnnotationTok}[1]{\textcolor[rgb]{0.56,0.35,0.01}{\textbf{\textit{#1}}}}
\newcommand{\AttributeTok}[1]{\textcolor[rgb]{0.77,0.63,0.00}{#1}}
\newcommand{\BaseNTok}[1]{\textcolor[rgb]{0.00,0.00,0.81}{#1}}
\newcommand{\BuiltInTok}[1]{#1}
\newcommand{\CharTok}[1]{\textcolor[rgb]{0.31,0.60,0.02}{#1}}
\newcommand{\CommentTok}[1]{\textcolor[rgb]{0.56,0.35,0.01}{\textit{#1}}}
\newcommand{\CommentVarTok}[1]{\textcolor[rgb]{0.56,0.35,0.01}{\textbf{\textit{#1}}}}
\newcommand{\ConstantTok}[1]{\textcolor[rgb]{0.00,0.00,0.00}{#1}}
\newcommand{\ControlFlowTok}[1]{\textcolor[rgb]{0.13,0.29,0.53}{\textbf{#1}}}
\newcommand{\DataTypeTok}[1]{\textcolor[rgb]{0.13,0.29,0.53}{#1}}
\newcommand{\DecValTok}[1]{\textcolor[rgb]{0.00,0.00,0.81}{#1}}
\newcommand{\DocumentationTok}[1]{\textcolor[rgb]{0.56,0.35,0.01}{\textbf{\textit{#1}}}}
\newcommand{\ErrorTok}[1]{\textcolor[rgb]{0.64,0.00,0.00}{\textbf{#1}}}
\newcommand{\ExtensionTok}[1]{#1}
\newcommand{\FloatTok}[1]{\textcolor[rgb]{0.00,0.00,0.81}{#1}}
\newcommand{\FunctionTok}[1]{\textcolor[rgb]{0.00,0.00,0.00}{#1}}
\newcommand{\ImportTok}[1]{#1}
\newcommand{\InformationTok}[1]{\textcolor[rgb]{0.56,0.35,0.01}{\textbf{\textit{#1}}}}
\newcommand{\KeywordTok}[1]{\textcolor[rgb]{0.13,0.29,0.53}{\textbf{#1}}}
\newcommand{\NormalTok}[1]{#1}
\newcommand{\OperatorTok}[1]{\textcolor[rgb]{0.81,0.36,0.00}{\textbf{#1}}}
\newcommand{\OtherTok}[1]{\textcolor[rgb]{0.56,0.35,0.01}{#1}}
\newcommand{\PreprocessorTok}[1]{\textcolor[rgb]{0.56,0.35,0.01}{\textit{#1}}}
\newcommand{\RegionMarkerTok}[1]{#1}
\newcommand{\SpecialCharTok}[1]{\textcolor[rgb]{0.00,0.00,0.00}{#1}}
\newcommand{\SpecialStringTok}[1]{\textcolor[rgb]{0.31,0.60,0.02}{#1}}
\newcommand{\StringTok}[1]{\textcolor[rgb]{0.31,0.60,0.02}{#1}}
\newcommand{\VariableTok}[1]{\textcolor[rgb]{0.00,0.00,0.00}{#1}}
\newcommand{\VerbatimStringTok}[1]{\textcolor[rgb]{0.31,0.60,0.02}{#1}}
\newcommand{\WarningTok}[1]{\textcolor[rgb]{0.56,0.35,0.01}{\textbf{\textit{#1}}}}
\usepackage{graphicx,grffile}
\makeatletter
\def\maxwidth{\ifdim\Gin@nat@width>\linewidth\linewidth\else\Gin@nat@width\fi}
\def\maxheight{\ifdim\Gin@nat@height>\textheight\textheight\else\Gin@nat@height\fi}
\makeatother
% Scale images if necessary, so that they will not overflow the page
% margins by default, and it is still possible to overwrite the defaults
% using explicit options in \includegraphics[width, height, ...]{}
\setkeys{Gin}{width=\maxwidth,height=\maxheight,keepaspectratio}
\setlength{\emergencystretch}{3em}  % prevent overfull lines
\providecommand{\tightlist}{%
  \setlength{\itemsep}{0pt}\setlength{\parskip}{0pt}}
\setcounter{secnumdepth}{0}
% Redefines (sub)paragraphs to behave more like sections
\ifx\paragraph\undefined\else
\let\oldparagraph\paragraph
\renewcommand{\paragraph}[1]{\oldparagraph{#1}\mbox{}}
\fi
\ifx\subparagraph\undefined\else
\let\oldsubparagraph\subparagraph
\renewcommand{\subparagraph}[1]{\oldsubparagraph{#1}\mbox{}}
\fi

% set default figure placement to htbp
\makeatletter
\def\fps@figure{htbp}
\makeatother


\title{Linear Algebra Chapter 1 Problems}
\author{Ryan Heslin}
\date{2021-11-17}

\begin{document}
\maketitle

\begin{Shaded}
\begin{Highlighting}[]
\KeywordTok{library}\NormalTok{(tidyverse)}
\KeywordTok{library}\NormalTok{(matador)}
\end{Highlighting}
\end{Shaded}

\hypertarget{section}{%
\section{1.1}\label{section}}

Pre-multiplying by a permutation matrix switches rows, post-multiplying
switches columns. Each 1 in the permutation ``slices'' every element of
a row (in the first case, since the 1 cuts across the row margin), or a
column (the second). Each column receives its own scalar because each
column represents a variable - a mutable quantity that can take on just
one value, but is assigned a scalar for each dimension (the numbers of
the matrix, reading downward.)

So as a transformation a vector serves as a collection of scalars for
variables, but itself contains only one variable able to be transformed.

\newcommand{\di}[2]{#1\times{#2}}
\newcommand{\m}[1]{\begin{bmatrix}#1\end{bmatrix}}

\(4\times{2}\)

Say we are asked to find a system with the solutions:

\[x=6+5t\\
y=4+3t\\
z=2+t\] Simply set up a matrix with the unknowns on the RHS and row
reduce to eliminate the \(t\) terms:

\[\begin{bmatrix}
t&2&z\\
3t&4&y\\
5t&6&x\end{bmatrix}\]

Row reduce:

\[\begin{bmatrix}
t&2&z\\
0&-2&y-3z\\
0&-4&x-5z\end{bmatrix}\] Then zero out the last row as usual to get the
last equation (the nullspace basis, as it happens:)

\[y-3z=-2\\
x-5z=-4\\
x-2y+z=0\]

\hypertarget{section-1}{%
\subsection{21.}\label{section-1}}

The three numbers whose sums in combinations of 2 are:

\begin{Shaded}
\begin{Highlighting}[]
\NormalTok{A <-}\StringTok{ }\KeywordTok{matrix}\NormalTok{(}\KeywordTok{rep}\NormalTok{(}\DecValTok{1}\NormalTok{, }\DecValTok{9}\NormalTok{), }\DataTypeTok{nrow =}\DecValTok{3}\NormalTok{)}
\KeywordTok{diag}\NormalTok{(A) <-}\StringTok{ }\DecValTok{0}

\KeywordTok{solve}\NormalTok{(A, }\KeywordTok{c}\NormalTok{(}\DecValTok{24}\NormalTok{, }\DecValTok{28}\NormalTok{, }\DecValTok{30}\NormalTok{))}
\end{Highlighting}
\end{Shaded}

\begin{verbatim}
## [1] 17 13 11
\end{verbatim}

\hypertarget{section-2}{%
\subsection{24.}\label{section-2}}

There are four possible leading one configurations for a \$ 3\times 2\$
RREF: none, \(1, 1\), \((1, 1)\) and \((2, 2)\), \((1, 1)\) and
\((2, 3\)) \# System of a Line

The system's solutions pass through (1,1,1) and (3,5,0). Since it is a
line, one variable is free. The coefficients on \(z\) are those that
satisfy both points on the line setting \(z\) at 1.
\[z\begin{bmatrix}3-2z\\5-4z\\z\end{bmatrix}\]

\hypertarget{section-3}{%
\subsection{34.}\label{section-3}}

Each output is the sum of the demand on each industry by each other
industry plus the consumer demand. the system is:

\[\begin{align}
&x_1=.2x_2+.3x_3+320\\
&x_2=.1x_1+.4x_3+90\\
&x_3=.2x_1+.5x_2+150
\end{align}\]

\begin{Shaded}
\begin{Highlighting}[]
\KeywordTok{solve}\NormalTok{(}\DataTypeTok{a =} \KeywordTok{matrix}\NormalTok{(}\KeywordTok{c}\NormalTok{(}\DecValTok{1}\NormalTok{, }\FloatTok{-.2}\NormalTok{, }\FloatTok{-.3}\NormalTok{, }
                   \FloatTok{-.1}\NormalTok{,}\DecValTok{1}\NormalTok{, }\FloatTok{-.4}\NormalTok{,}
                   \FloatTok{-.2}\NormalTok{, }\FloatTok{-.5}\NormalTok{, }\DecValTok{1}\NormalTok{), }\DataTypeTok{byrow =} \OtherTok{TRUE}\NormalTok{, }\DataTypeTok{nrow =} \DecValTok{3}\NormalTok{),}
      \DataTypeTok{b =} \KeywordTok{matrix}\NormalTok{(}\KeywordTok{c}\NormalTok{(}\DecValTok{320}\NormalTok{, }\DecValTok{90}\NormalTok{, }\DecValTok{150}\NormalTok{)))}
\end{Highlighting}
\end{Shaded}

\begin{verbatim}
##      [,1]
## [1,]  500
## [2,]  300
## [3,]  400
\end{verbatim}

\hypertarget{section-4}{%
\subsection{35.}\label{section-4}}

Polynomials of the form \(8 + t -8t^2\)

\begin{Shaded}
\begin{Highlighting}[]
\KeywordTok{solve}\NormalTok{(}\DataTypeTok{a =} \KeywordTok{square}\NormalTok{(}\DecValTok{1}\NormalTok{, }\DecValTok{1}\NormalTok{, }\DecValTok{1}\NormalTok{, }\DecValTok{1}\NormalTok{, }\DecValTok{3}\NormalTok{, }\DecValTok{9}\NormalTok{, }\DecValTok{1}\NormalTok{, }\DecValTok{1}\NormalTok{, }\DecValTok{2}\NormalTok{),}
      \DataTypeTok{b =} \KeywordTok{c}\NormalTok{(}\DecValTok{1}\NormalTok{, }\DecValTok{3}\NormalTok{, }\DecValTok{1}\NormalTok{))}
\end{Highlighting}
\end{Shaded}

\begin{verbatim}
## [1]  8  1 -8
\end{verbatim}

\hypertarget{section-5}{%
\subsection{37.}\label{section-5}}

\begin{Shaded}
\begin{Highlighting}[]
\KeywordTok{solve}\NormalTok{(}\DataTypeTok{a =} \KeywordTok{square}\NormalTok{(}\DecValTok{1}\NormalTok{, }\DecValTok{1}\NormalTok{, }\DecValTok{3}\NormalTok{, }\DecValTok{2}\NormalTok{), }\DataTypeTok{b =} \KeywordTok{c}\NormalTok{(}\DecValTok{1}\NormalTok{, }\DecValTok{4}\NormalTok{))}
\end{Highlighting}
\end{Shaded}

\begin{verbatim}
## [1] 10 -3
\end{verbatim}

\hypertarget{section-6}{%
\subsection{45.}\label{section-6}}

A system in three unknowns with solutions \((6+5t, 4 + 3t, 2 +t)\)

\[
  \begin{aligned}
    & 2x + y -z = 18 + 14 t\\
    & x +y + z = 12 + 9t\\
    & 3x -y + 2z = 16+ 14t
  \end{aligned}
\] \# 1.2

\hypertarget{section-7}{%
\subsection{19.}\label{section-7}}

All \(4 \times 1\) RREFS are either
\(\begin{bmatrix}1\\0\\0\\0\end{bmatrix}\) or
\(\begin{bmatrix}0\\0\\0\\0\end{bmatrix}\). \#\# 20.

Given the RREF of

\[
  \begin{bmatrix}
  0 & a & 2 &1 & b\\
  0 & 0 &0 & c &d\\
  0 & 0 & e & 0 &0
  \end{bmatrix}
\] \(a\) must be nonzero, \(b\) is arbitrary, \(c\) is 0, \(e\) must be
0. \(d\) is either 1 or 0; in the latter case so is \(b\).

\hypertarget{section-8}{%
\subsection{21.}\label{section-8}}

Given the RREF of

\[
  \begin{bmatrix}
  1 & a& b & 3 & 0  & -2\\
  0 & 0 & c & 1 & d & 3\\
  0 & e & 0 & 0 & 1 & 1
  \end{bmatrix}
\]

\(e\) must be 0, \(a\) and \(b\) are arbitrary, though \(b\) is 0 if
\(c\), also arbitrary, and \(d\) must be 0.

\hypertarget{section-9}{%
\subsection{32.}\label{section-9}}

Finding a cubic: It turns out to be \(a +2t -t^2 -2t^3\).

\begin{Shaded}
\begin{Highlighting}[]
\NormalTok{A <-}\StringTok{ }\KeywordTok{cbind}\NormalTok{(}\KeywordTok{rep}\NormalTok{(}\DecValTok{1}\NormalTok{, }\DecValTok{4}\NormalTok{),}
           \KeywordTok{matrix}\NormalTok{(}\KeywordTok{c}\NormalTok{(}\DecValTok{0}\NormalTok{, }\DecValTok{0}\NormalTok{, }\DecValTok{0}\NormalTok{, }\DecValTok{1}\NormalTok{, }\DecValTok{1}\NormalTok{, }\DecValTok{1}\NormalTok{, }\DecValTok{-1}\NormalTok{, }\DecValTok{1}\NormalTok{, }\DecValTok{-1}\NormalTok{, }\DecValTok{2}\NormalTok{, }\DecValTok{4}\NormalTok{, }\DecValTok{8}\NormalTok{), }\DataTypeTok{nrow =} \DecValTok{4}\NormalTok{, }\DataTypeTok{byrow =} \OtherTok{TRUE}\NormalTok{))}
\NormalTok{b <-}\StringTok{ }\KeywordTok{c}\NormalTok{(}\DecValTok{1}\NormalTok{, }\DecValTok{0}\NormalTok{, }\DecValTok{0}\NormalTok{, }\DecValTok{-15}\NormalTok{)}
\KeywordTok{solve}\NormalTok{(A, b)}
\end{Highlighting}
\end{Shaded}

\begin{verbatim}
## [1]  1  2 -1 -2
\end{verbatim}

\hypertarget{section-10}{%
\subsection{35.}\label{section-10}}

Now for a derivative example. \(f`(t) = b + 2ct +3dt^2\)

It turns out to be \(.625a - 1.0625bt + 1.25ct^2 + .1875dt^3\).

\begin{Shaded}
\begin{Highlighting}[]
\NormalTok{A <-}\StringTok{ }\KeywordTok{cbind}\NormalTok{(}\KeywordTok{c}\NormalTok{(}\DecValTok{1}\NormalTok{, }\DecValTok{1}\NormalTok{, }\DecValTok{0}\NormalTok{, }\DecValTok{0}\NormalTok{), }
           \KeywordTok{matrix}\NormalTok{(}\KeywordTok{c}\NormalTok{(}\DecValTok{1}\NormalTok{, }\DecValTok{1}\NormalTok{, }\DecValTok{1}\NormalTok{, }\DecValTok{2}\NormalTok{, }\DecValTok{4}\NormalTok{, }\DecValTok{8}\NormalTok{, }\DecValTok{1}\NormalTok{, }\DecValTok{2}\NormalTok{, }\DecValTok{3}\NormalTok{, }\DecValTok{1}\NormalTok{, }\DecValTok{4}\NormalTok{, }\DecValTok{27}\NormalTok{), }\DataTypeTok{nrow =} \DecValTok{4}\NormalTok{, }\DataTypeTok{byrow =} \OtherTok{TRUE}\NormalTok{))}
\NormalTok{b <-}\StringTok{ }\KeywordTok{c}\NormalTok{(}\DecValTok{1}\NormalTok{, }\DecValTok{5}\NormalTok{, }\DecValTok{2}\NormalTok{, }\DecValTok{9}\NormalTok{)}
\KeywordTok{solve}\NormalTok{(A, b)}
\end{Highlighting}
\end{Shaded}

\begin{verbatim}
## [1]  0.6250 -1.0625  1.2500  0.1875
\end{verbatim}

\hypertarget{section-11}{%
\subsection{45.}\label{section-11}}

We plug the given values into the formula for \(t\) and use \(S(t\) as
the solution vector \(b\). Then just solve the system:

\begin{Shaded}
\begin{Highlighting}[]
\KeywordTok{library}\NormalTok{(tidyverse)}
\NormalTok{t <-}\StringTok{ }\KeywordTok{c}\NormalTok{(}\DecValTok{47}\NormalTok{, }\DecValTok{74}\NormalTok{, }\DecValTok{273}\NormalTok{)}
\NormalTok{s_t <-}\StringTok{ }\KeywordTok{matrix}\NormalTok{(}\KeywordTok{c}\NormalTok{(}\FloatTok{11.5}\NormalTok{, }\DecValTok{12}\NormalTok{, }\DecValTok{12}\NormalTok{))}
\NormalTok{mat <-}\StringTok{ }\KeywordTok{cbind}\NormalTok{(}\KeywordTok{c}\NormalTok{(}\DecValTok{1}\NormalTok{,}\DecValTok{1}\NormalTok{,}\DecValTok{1}\NormalTok{), }
             \KeywordTok{cos}\NormalTok{(((}\DecValTok{2}\OperatorTok{*}\NormalTok{pi) }\OperatorTok{*}\StringTok{ }\NormalTok{t ) }\OperatorTok{/}\StringTok{ }\DecValTok{365}\NormalTok{),}
             \KeywordTok{sin}\NormalTok{(((}\DecValTok{2}\OperatorTok{*}\NormalTok{pi) }\OperatorTok{*}\StringTok{ }\NormalTok{t ) }\OperatorTok{/}\StringTok{ }\DecValTok{365}\NormalTok{))}

\NormalTok{coefs <-}\StringTok{ }\KeywordTok{solve}\NormalTok{(}\DataTypeTok{a =}\NormalTok{ mat, }\DataTypeTok{b=}\NormalTok{ s_t)}


\NormalTok{days <-}\StringTok{ }\KeywordTok{tibble}\NormalTok{( }\DataTypeTok{day =} \DecValTok{1}\OperatorTok{:}\DecValTok{365}\NormalTok{, }\DataTypeTok{len =}\NormalTok{ coefs[}\DecValTok{1}\NormalTok{]}\OperatorTok{+}\StringTok{ }\NormalTok{(coefs[}\DecValTok{2}\NormalTok{] }\OperatorTok{*}\StringTok{ }\KeywordTok{cos}\NormalTok{((}\DecValTok{2}\OperatorTok{*}\NormalTok{pi}\OperatorTok{*}\DecValTok{1}\OperatorTok{:}\DecValTok{365}\NormalTok{)}\OperatorTok{/}\StringTok{ }\DecValTok{365}\NormalTok{)) }\OperatorTok{+}\StringTok{ }\NormalTok{(coefs[}\DecValTok{3}\NormalTok{] }\OperatorTok{*}\StringTok{ }\KeywordTok{sin}\NormalTok{((}\DecValTok{2}\OperatorTok{*}\NormalTok{pi}\OperatorTok{*}\DecValTok{1}\OperatorTok{:}\DecValTok{365}\NormalTok{)}\OperatorTok{/}\StringTok{ }\DecValTok{365}\NormalTok{)))}

\KeywordTok{ggplot}\NormalTok{(days, }\KeywordTok{aes}\NormalTok{(day, len)) }\OperatorTok{+}
\StringTok{  }\KeywordTok{geom_line}\NormalTok{() }\OperatorTok{+}
\StringTok{  }\KeywordTok{geom_vline}\NormalTok{(}\KeywordTok{aes}\NormalTok{(}\DataTypeTok{xintercept =}\NormalTok{ day[}\KeywordTok{which.max}\NormalTok{(len)]))}
\end{Highlighting}
\end{Shaded}

\includegraphics{/home/rheslin/R/Projects/spring_21/linalg/problems/markdowns/practice_ch1_files/figure-latex/unnamed-chunk-8-1.pdf}

\begin{Shaded}
\begin{Highlighting}[]
\KeywordTok{max}\NormalTok{(days}\OperatorTok{$}\NormalTok{len)               }
\end{Highlighting}
\end{Shaded}

\begin{verbatim}
## [1] 13.33135
\end{verbatim}

The longest day is about 13 hours long. \#\# 47.

Infintie: \(k=2\) Inconsistent: \(k= 1\) Otherwise: unique

\hypertarget{section-12}{%
\subsection{48.}\label{section-12}}

By row reducing the system to:

\[\begin{bmatrix}1&2&3&4\\
0&k-2&1&2\\
0&0&k-1&2\end{bmatrix}\]

we see that it inconsistent if \(k=1\), infinitely many solutions if
\(k=3\), and a unique solution otherwise.

\hypertarget{section-13}{%
\subsection{66.}\label{section-13}}

There were 120 liberals and 140 conservatives at the start, 140 and 120
respectively at the end.

\begin{Shaded}
\begin{Highlighting}[]
\KeywordTok{solve}\NormalTok{(matador}\OperatorTok{::}\KeywordTok{square}\NormalTok{(}\DecValTok{1}\NormalTok{, }\DecValTok{1}\NormalTok{, }\DecValTok{0}\NormalTok{, }\DecValTok{0}\NormalTok{, }\FloatTok{-.3}\NormalTok{, }\FloatTok{-.6}\NormalTok{, }\DecValTok{0}\NormalTok{, }\DecValTok{1}\NormalTok{, }\FloatTok{-.7}\NormalTok{, }\FloatTok{-.4}\NormalTok{, }\DecValTok{1}\NormalTok{, }\DecValTok{0}\NormalTok{, }\DecValTok{1}\NormalTok{, }\DecValTok{0}\NormalTok{, }\DecValTok{0}\NormalTok{, }\DecValTok{-1}\NormalTok{, }\DataTypeTok{byrow =} \OtherTok{TRUE}\NormalTok{), }\DataTypeTok{b =} \KeywordTok{c}\NormalTok{(}\DecValTok{260}\NormalTok{, }\DecValTok{0}\NormalTok{, }\DecValTok{0}\NormalTok{, }\DecValTok{0}\NormalTok{))}
\end{Highlighting}
\end{Shaded}

\begin{verbatim}
## [1] 120 140 140 120
\end{verbatim}

\hypertarget{section-14}{%
\subsection{74.}\label{section-14}}

\begin{Shaded}
\begin{Highlighting}[]
\NormalTok{((}\DecValTok{1500}\OperatorTok{/}\DecValTok{19}\NormalTok{) }\OperatorTok{-}\NormalTok{(}\DecValTok{15}\OperatorTok{/}\DecValTok{19}\NormalTok{) }\OperatorTok{*}\DecValTok{1}\OperatorTok{:}\DecValTok{100}\NormalTok{)}
\end{Highlighting}
\end{Shaded}

\begin{verbatim}
##   [1] 78.1578947 77.3684211 76.5789474 75.7894737
##   [5] 75.0000000 74.2105263 73.4210526 72.6315789
##   [9] 71.8421053 71.0526316 70.2631579 69.4736842
##  [13] 68.6842105 67.8947368 67.1052632 66.3157895
##  [17] 65.5263158 64.7368421 63.9473684 63.1578947
##  [21] 62.3684211 61.5789474 60.7894737 60.0000000
##  [25] 59.2105263 58.4210526 57.6315789 56.8421053
##  [29] 56.0526316 55.2631579 54.4736842 53.6842105
##  [33] 52.8947368 52.1052632 51.3157895 50.5263158
##  [37] 49.7368421 48.9473684 48.1578947 47.3684211
##  [41] 46.5789474 45.7894737 45.0000000 44.2105263
##  [45] 43.4210526 42.6315789 41.8421053 41.0526316
##  [49] 40.2631579 39.4736842 38.6842105 37.8947368
##  [53] 37.1052632 36.3157895 35.5263158 34.7368421
##  [57] 33.9473684 33.1578947 32.3684211 31.5789474
##  [61] 30.7894737 30.0000000 29.2105263 28.4210526
##  [65] 27.6315789 26.8421053 26.0526316 25.2631579
##  [69] 24.4736842 23.6842105 22.8947368 22.1052632
##  [73] 21.3157895 20.5263158 19.7368421 18.9473684
##  [77] 18.1578947 17.3684211 16.5789474 15.7894737
##  [81] 15.0000000 14.2105263 13.4210526 12.6315789
##  [85] 11.8421053 11.0526316 10.2631579  9.4736842
##  [89]  8.6842105  7.8947368  7.1052632  6.3157895
##  [93]  5.5263158  4.7368421  3.9473684  3.1578947
##  [97]  2.3684211  1.5789474  0.7894737  0.0000000
\end{verbatim}

\begin{Shaded}
\begin{Highlighting}[]
\NormalTok{(}\DecValTok{400}\OperatorTok{/}\DecValTok{19}\NormalTok{) }\OperatorTok{-}\StringTok{ }\DecValTok{1}\OperatorTok{:}\DecValTok{100}
\end{Highlighting}
\end{Shaded}

\begin{verbatim}
##   [1]  20.05263158  19.05263158  18.05263158
##   [4]  17.05263158  16.05263158  15.05263158
##   [7]  14.05263158  13.05263158  12.05263158
##  [10]  11.05263158  10.05263158   9.05263158
##  [13]   8.05263158   7.05263158   6.05263158
##  [16]   5.05263158   4.05263158   3.05263158
##  [19]   2.05263158   1.05263158   0.05263158
##  [22]  -0.94736842  -1.94736842  -2.94736842
##  [25]  -3.94736842  -4.94736842  -5.94736842
##  [28]  -6.94736842  -7.94736842  -8.94736842
##  [31]  -9.94736842 -10.94736842 -11.94736842
##  [34] -12.94736842 -13.94736842 -14.94736842
##  [37] -15.94736842 -16.94736842 -17.94736842
##  [40] -18.94736842 -19.94736842 -20.94736842
##  [43] -21.94736842 -22.94736842 -23.94736842
##  [46] -24.94736842 -25.94736842 -26.94736842
##  [49] -27.94736842 -28.94736842 -29.94736842
##  [52] -30.94736842 -31.94736842 -32.94736842
##  [55] -33.94736842 -34.94736842 -35.94736842
##  [58] -36.94736842 -37.94736842 -38.94736842
##  [61] -39.94736842 -40.94736842 -41.94736842
##  [64] -42.94736842 -43.94736842 -44.94736842
##  [67] -45.94736842 -46.94736842 -47.94736842
##  [70] -48.94736842 -49.94736842 -50.94736842
##  [73] -51.94736842 -52.94736842 -53.94736842
##  [76] -54.94736842 -55.94736842 -56.94736842
##  [79] -57.94736842 -58.94736842 -59.94736842
##  [82] -60.94736842 -61.94736842 -62.94736842
##  [85] -63.94736842 -64.94736842 -65.94736842
##  [88] -66.94736842 -67.94736842 -68.94736842
##  [91] -69.94736842 -70.94736842 -71.94736842
##  [94] -72.94736842 -73.94736842 -74.94736842
##  [97] -75.94736842 -76.94736842 -77.94736842
## [100] -78.94736842
\end{verbatim}

15 sparrows, 6 ducks, 79 roosters.

\hypertarget{section-15}{%
\subsection{50.}\label{section-15}}

By row elimination and induction, we find the solution (where \(k=n-1\))
is \(x_k =(1+n-k)x_n-x_{n+1}\)

\hypertarget{section-16}{%
\subsection{79.}\label{section-16}}

They have 16, 12, and 8 coins, respectively.

\begin{Shaded}
\begin{Highlighting}[]
\NormalTok{A <-}\StringTok{ }\NormalTok{matador}\OperatorTok{::}\KeywordTok{square}\NormalTok{(}\DecValTok{1}\NormalTok{, }\DecValTok{-2}\NormalTok{, }\DecValTok{-2}\NormalTok{,}
                     \DecValTok{-3}\NormalTok{, }\DecValTok{1}\NormalTok{, }\DecValTok{-3}\NormalTok{,}
                     \DecValTok{-5}\NormalTok{, }\DecValTok{-5}\NormalTok{, }\DecValTok{1}\NormalTok{)}

\KeywordTok{solve}\NormalTok{(A, }\DataTypeTok{b =} \KeywordTok{setNames}\NormalTok{(}\KeywordTok{rep}\NormalTok{(}\OperatorTok{-}\DecValTok{60}\NormalTok{, }\DecValTok{3}\NormalTok{), }\DecValTok{1}\OperatorTok{:}\DecValTok{3}\NormalTok{))}
\end{Highlighting}
\end{Shaded}

\begin{verbatim}
## [1] 16 12  8
\end{verbatim}

\hypertarget{section-17}{%
\subsection{80.}\label{section-17}}

\begin{Shaded}
\begin{Highlighting}[]
\NormalTok{A <-}\StringTok{ }\NormalTok{matador}\OperatorTok{::}\KeywordTok{square}\NormalTok{(}\OperatorTok{-}\DecValTok{1}\NormalTok{, }\DecValTok{2}\NormalTok{, }\DecValTok{-3}\NormalTok{,}
                     \DecValTok{-4}\NormalTok{, }\DecValTok{4}\NormalTok{, }\DecValTok{-7}\NormalTok{,}
                     \DecValTok{-2}\NormalTok{, }\DecValTok{5}\NormalTok{, }\DecValTok{-3}\NormalTok{)}
\KeywordTok{solve}\NormalTok{(A)}
\end{Highlighting}
\end{Shaded}

\begin{verbatim}
##            [,1]       [,2]        [,3]
## [1,]  1.3529412  0.1176471 -0.70588235
## [2,] -0.5294118 -0.1764706  0.05882353
## [3,] -0.1176471  0.2941176  0.23529412
\end{verbatim}

\hypertarget{section-18}{%
\section{1.3}\label{section-18}}

\hypertarget{section-19}{%
\subsection{29-32.}\label{section-19}}

Let \(x=\begin{bmatrix}5\\3\\-9\end{bmatrix}\) and
\(b=\begin{bmatrix}2\\0\\1\end{bmatrix}\). To find \(A\) from these
vectors, we just solve the system of equations \[5a_1+3a_2-9a_3=b_n\\
a_1=\frac{b_2-3a_2-9a_3}{5}\] substituting each element of \(A\) for the
corresponding row of \(A\). The rank 1 \(A\):

\[\begin{bmatrix}2/5&0&0\\
0&0&0\\
1/5&0&0\end{bmatrix}\]

All-nonzero \(A\), by solving the equation above for each \(b_n\) and
subbing 1 for free variables,

\[A=\begin{bmatrix}5&1&1\\
6/5&1&1\\
7/5&1&1\end{bmatrix}\]

\hypertarget{section-20}{%
\subsection{46.}\label{section-20}}

The rank of

\[A=\begin{bmatrix}1&b&c\\
0&d&e\\
0&0&f\end{bmatrix}\] where the diagonals are nonzero, the others
arbitrary. This matrix has rank

\hypertarget{section-21}{%
\subsection{49.}\label{section-21}}

Ranks given some information about matrices:

3x4 augment rank 2: rank 1 or 2. Partial row and column rank, so either
no o infinitely many solutions.

\begin{enumerate}
\def\labelenumi{\alph{enumi}.}
\setcounter{enumi}{2}
\tightlist
\item
  \(4\times{3}\), augment rank 4: no solution, since \(b\) increases
  rank. \(4\times{3}\) rank 3 has augment rank 3 or 4,d depending on
  whether \(Ax=b\) has a solution. Partial row rank and full column
  rank, so either one or zero solutions.
\end{enumerate}

3 x4, rank 3: augment rank 3, since full row rank guarantees a solution,
so \(AX=b\) for all \(b\) Infinitely many solutions. \#\# Costs and
Units

A system where the total number of items and their individual values and
total value are known. Solutions are constrained to be integers (no
fractional items).

For example, 32 \$1, \$5, and \$10 bills worth \$100 total. The system
is: \[x+y+z=32\\
x+5y+10z=100\] Row reduce to get \[x = 5/4z+15\\
y=17-9/4z\]

since the variables must be integers and \(10z<=100\), \(z\) can only be
4 or 8. substituting we see it is 8:

\[20/4+15+17-9/4+4=32\\
32=32\]

If the \(\4\times{4}\) \(Ax=b\) has a unique solution, then \(Ax=c\) has
either one or none, since the matrix has full row rank (for a unique
solution) and therefor full column rank.

If \(Ax=b\) is consistent, both column and row rank are partial, and
\(Ax=c\) has either one or infinite solutions.

if \(A\) is \(4\times{3}\) and \(Ax=b\) has a unique solution, \(Ax=c\)
has either a unique solution or none. Rank is 3 (unique solution for 3
columns), but partial row rank does not guarantee solutions.

For rows, full rank means the guarantee of a solution; for columns, of a
\emph{unique} solution.

If \(rank(A)=n\) and \(m>n\), \(m-n\) rows are zeroed. If
\(rank(A) = m\) and \(n>m\), then \(n-m\) free variables remain.

\begin{enumerate}
\def\labelenumi{\arabic{enumi}.}
\setcounter{enumi}{26}
\item
  A rank-4 \(4\times{4}\) has the identity as RREf.
\item
  A \(5\times{3}\) matrix is 3, its RREF has two zeroed bottom rows.
\item
  Prove a tall matrix is always inconsistent for some \(b\).\\
  Since any \(k>n\) vectors in \(R^n\) are linearly dependent, an
  \(m >n\) matrix has linearly dependent rows. Therefore each row can be
  expressed \(r_i=c_1r_j+c_2r_k+\dotsc_nr_l\) Let \(b_i\) be a linear
  combination of \(b\) with different scalars such that
  \(b_i\neq{c_1}b_j+c_2b_k+\dots{c_n}b_l\). Since \(b\) and \(A\) are
  independently determined, this is always possible.
\item
  Prove \(A(kx) = k(Ax)\). Very easy. Referring to column vectors:
\end{enumerate}

\[A(kx) = \begin{bmatrix}kv_1+kv_2+\dots kv_n\end{bmatrix}\]
\[k(Ax) = k\begin{bmatrix}kv_1+kv_2+\dots kv_n\end{bmatrix}=
\begin{bmatrix}kv_1+kv_2+\dots kv_n\end{bmatrix}\]

\hypertarget{section-22}{%
\subsection{46.}\label{section-22}}

Find the rank of \[\begin{bmatrix}a&b&c\\
0&d&e\\
0&0&f\end{bmatrix}\] assuming nonzero diagonal.

Rank is 3. Every row is a pivot row, so \(b\), \(c\), and \(e\) can be
eliminated, whatever they are, and the rows scaled to yield the
identity. 47. If a system is homogeneous:

\begin{enumerate}
\def\labelenumi{\arabic{enumi}.}
\tightlist
\item
  It is never inconsistent because
  \(c_1r_1+c_2r_2+c_nr_n=0c_1+0c_2+0c_n\). No matter what LC reduces
  rows to 0, all LCs of the zero vector are 0.
\end{enumerate}

\begin{enumerate}
\def\labelenumi{\alph{enumi}.}
\setcounter{enumi}{1}
\item
  A consistent partial column rank matrix always has solutions. Since
  homogeneous systems are always consistent.
\item
  \(x_1+x_2=0+0\). If the equation has nullspace (and therefore
  infinitely many solutions), any LC of a given \(x\) is a solution.
\item
  True also of \(Ax=0\rightarrow{Akx=0}\)
\end{enumerate}

\begin{enumerate}
\def\labelenumi{\arabic{enumi}.}
\setcounter{enumi}{47}
\item
  \begin{enumerate}
  \def\labelenumii{\alph{enumii}.}
  \tightlist
  \item
    In general, if \(x_a\) and \(x_b\) lie in the nullspace, then
    \(x_a+x_b\) does as well, since the nullspace contains all LC's of
    the basis vectors.
  \end{enumerate}
\end{enumerate}

\begin{enumerate}
\def\labelenumi{\alph{enumi}.}
\setcounter{enumi}{1}
\item
  Also, if \(x_b=x_a\), then we have \(A0=0\)
\item
  \(\begin{bmatrix}x_1-x_2//0\end{bmatrix}\). The dependent row is
  zeroed.
\end{enumerate}

\begin{enumerate}
\def\labelenumi{\arabic{enumi}.}
\setcounter{enumi}{34}
\item
  \(A_e_i\). The identity matrix copies each input at its position
\item
  \(A\) 4x3, augment has rank 4. There are is a unique solution because
  adding \(b\) increases the rank, so no linear dependency is created.
\item
  For \(m\times{n}\) A and \(r\times{s}\), and \(x\) in \(R^p\),
  \(A(Bx)\) is defined if \(s =p\) and \(n=r\), since \(Bx\) has
  \(B's rows\)
\item
  If two \(R^3\) vectors are nonparallel, they span \(c_1v_1+c_2v_2\)
  for all real \(c\). They always span the zero vector, and span either
  a line or plane, depending on linear dependency.
\end{enumerate}

\hypertarget{section-23}{%
\subsection{52.}\label{section-23}}

Just multiply them together.

\hypertarget{section-24}{%
\subsection{54.}\label{section-24}}

Two nonparallel vectors in \(R^3\) describe a plane.

\hypertarget{section-25}{%
\subsection{57.}\label{section-25}}

To get a vector as an LC of two lines, pick vectors on the lines and
solve:

\begin{Shaded}
\begin{Highlighting}[]
\KeywordTok{solve}\NormalTok{(}\DataTypeTok{a =} \KeywordTok{matrix}\NormalTok{(}\KeywordTok{c}\NormalTok{(}\DecValTok{1}\NormalTok{, }\DecValTok{3}\NormalTok{, }\DecValTok{2}\NormalTok{, }\DecValTok{1}\NormalTok{), }\DataTypeTok{nrow =} \DecValTok{2}\NormalTok{), }\DataTypeTok{b =} \KeywordTok{matrix}\NormalTok{(}\KeywordTok{c}\NormalTok{(}\DecValTok{7}\NormalTok{,}\DecValTok{11}\NormalTok{)))}
\end{Highlighting}
\end{Shaded}

\begin{verbatim}
##      [,1]
## [1,]    3
## [2,]    2
\end{verbatim}

\hypertarget{section-26}{%
\subsection{59.}\label{section-26}}

The system has a solution at \(c=9, d= 11\)

\begin{Shaded}
\begin{Highlighting}[]
\KeywordTok{matrix}\NormalTok{(}\KeywordTok{c}\NormalTok{(}\KeywordTok{rep}\NormalTok{(}\DecValTok{1}\NormalTok{, }\DecValTok{4}\NormalTok{), }\DecValTok{1}\OperatorTok{:}\DecValTok{4}\NormalTok{), }\DataTypeTok{nrow =} \DecValTok{4}\NormalTok{) }\OperatorTok\StringTok{ }\KeywordTok{matrix}\NormalTok{(}\KeywordTok{c}\NormalTok{(}\DecValTok{3}\NormalTok{, }\DecValTok{2}\NormalTok{))}
\end{Highlighting}
\end{Shaded}

\begin{verbatim}
##      [,1]
## [1,]    5
## [2,]    7
## [3,]    9
## [4,]   11
\end{verbatim}

\hypertarget{section-27}{%
\subsection{63.}\label{section-27}}

All vectors of the form \(v +cw\) are shears - distortions that add some
scalar of one vector to another, such that the origianl vector remaisn
at the center. Assuming this order, vertical shears.

\hypertarget{section-28}{%
\subsection{65.}\label{section-28}}

Transformations of the form \(av+ bw\) are dual shears, both horizontal
and vertical.

\hypertarget{section-29}{%
\subsection{66.}\label{section-29}}

Transformations of this form where \(a +b =1\) are projections of
positive portions of \(v\) and \(w\) onto one another.

\hypertarget{section-30}{%
\subsection{67.}\label{section-30}}

If we add the constraint \(a +b=\), we have rotations.

\hypertarget{section-31}{%
\subsection{68.}\label{section-31}}

\(u \cdot v =u \cdot w\) for all vectors \(u\) that bisect \(v\) and
\(w\) and have half the length(I think)
\[u_1(v_1 - w_2) = u_2(v_2 - w_2)\]

\hypertarget{true-or-false}{%
\section{True or False?}\label{true-or-false}}

\begin{enumerate}
\def\labelenumi{\arabic{enumi}.}
\item
  True; \(Ax\) is a linear combination of the columns and \(x\).
\item
  True also for vectors.
\item
  True; no more rows can be eliminated.
\end{enumerate}

4 False; a \(4\times{3}\) system is consistent for \(x=0\) at least.

\begin{enumerate}
\def\labelenumi{\arabic{enumi}.}
\setcounter{enumi}{4}
\item
  False; no \(3\times{4}\) matrix can have rank above 3.
\item
  True; the the product has dimension of the LHS.
\item
  False; many square matrices are not one-to-one.
\item
  False; a system only ever has zero, one, or infinite solutions.
\item
  False; a \(rank <m\) matrix always has null space beyond the zero
  vector because it has at least one free variables.
\item
  False; columns of free variables are not reduced to ones.
\item
  True; a row of zeroes equal to anything besides 0 always means
  inconsistency.
\item
  True; in general, a diagonal matrix can be found for any \(b\) given
  any \(x\) that just scales each element appropriately.
\item
  True, assuming \(a\neq{b}\)
\item
  True; these vectors are linearly independent.
\item
  False. It could be something like:
\end{enumerate}

\[\begin{bmatrix}
0&0&0\\
0&0&0\\
0&0&0\\
0&0&1\end{bmatrix}\]

and \[x=\m{0\\0\\1\]

\begin{enumerate}
\def\labelenumi{\arabic{enumi}.}
\setcounter{enumi}{15}
\item
  False. \(kAx=A(kx)\), so if \(Ax=\begin{bmatrix}1\\2\end{bmatrix}\),
  then \(A(2x)\) should be \(\begin{bmatrix}2\\4\end{bmatrix}\), not
  \(\begin{bmatrix}2\\1\end{bmatrix}\).
\item
  False. All linear combinations of linearly dependent vectors are
  themselves linearly dependent.
\item
  False; a matrix consisting only of one value only ever has rank one.
\item
  False. The product has two dimensions.
\end{enumerate}

\begin{Shaded}
\begin{Highlighting}[]
\KeywordTok{matrix}\NormalTok{(}\KeywordTok{c}\NormalTok{(}\DecValTok{11}\NormalTok{, }\DecValTok{17}\NormalTok{,}\DecValTok{13}\NormalTok{,}\DecValTok{19}\NormalTok{, }\DecValTok{15}\NormalTok{, }\DecValTok{21}\NormalTok{), }\DataTypeTok{nrow =} \DecValTok{2}\NormalTok{) }\OperatorTok\StringTok{ }\KeywordTok{matrix}\NormalTok{(}\KeywordTok{c}\NormalTok{(}\OperatorTok{-}\DecValTok{1}\NormalTok{, }\DecValTok{3}\NormalTok{, }\DecValTok{-1}\NormalTok{))}
\end{Highlighting}
\end{Shaded}

\begin{verbatim}
##      [,1]
## [1,]   13
## [2,]   19
\end{verbatim}

\begin{enumerate}
\def\labelenumi{\arabic{enumi}.}
\setcounter{enumi}{19}
\item
  True; \(x_1=2\) and \(x_2=1\).
\item
  True. One such matrix is:
\end{enumerate}

\[\begin{bmatrix}-1&1\\
-3&1\\
-5&1\end{bmatrix}\]

\begin{enumerate}
\def\labelenumi{\arabic{enumi}.}
\setcounter{enumi}{19}
\item
  True; \(x=\begin{bmatrix}2\\-1\end{bmatrix}\)
\item
  False; a tall matrix has unique solutions for some \(b\), though it is
  not consistent for all.
\item
  False. A tall matrix with full column but partial row rank will have
  unique solutions for some \(b\) (and none for others).
\item
  True; a tall matrix is always inconsistent for some \(b\) if \(b\) is
  not the same linear combination of the basis vectors as the rows.
  Since a matrix of partial row rank, which a \(4\times{3}\) is, does
  not span its column space, such a vector always exists.
\item
  False. Since the unknowns in the upper and lower triangles just have
  opposite signs, the system would be linearly dependent.
\item
  False; if \(x\) and \(w\) are \(R_4\) vectors, \(v\) is only an LC of
  \(w\) if they are coplanar or collinear and therefore in each others'
  span.
\item
  If \(u\), \(v\), and \(w\) are nonzero \(R^2\) vectors, \(W\) is only
  an LC of the others if all three are collinear or \(u\) and \(v\) are
  linearly independent, since collinear vectors have the same span.
\item
  True; the zero vector is a linear combination of any and all vectors.
\item
  True. Normally row operations can't handle column permutations, but in
  a matrix with partial row rank every row is a linear combination of
  the others, so you can convert between any rank-2 matrices by row ops
  alone.
\item
  If \(c_1v+c_2w=u\) and \(c_1p+c_2q+c_3r=v\), then \$c\_1()
\item
  False; it must have either one or none.
\item
  True; every pivot column gets a nonzero entry.
\item
  False. If the 4x3 has full column rank, it has only unique solutions
  and therefore no null space and therefore no nontrivial solutions to
  \(Ax=0\).
\item
  True, obviously.
\item
  True. This matrix has full column rank, so every \(b\) has a unique
  \(x\).
\item
  True. If \(A\) is square and has a unique solution for any \(b\), it
  must have full rank, meaning \(Ax=0\) has only the trivial solution.
\item
  True.
\end{enumerate}

\[c_1v+c_2w=u\\
c_2w = u-c_1v\\
w = \frac{u-c_1v}{c_2}\]

\begin{enumerate}
\def\labelenumi{\arabic{enumi}.}
\setcounter{enumi}{35}
\item
  True. The values in a free variable's column represent that column as
  a linear combination of the other column vectors.
\item
  Obviously false. Adding two rank-3 \(3\times{3}\)s will not give you a
  rank 6!
\item
  False. Row ops can do anything except create leading entries, so as
  long as rank is the same, any two full-rank square matrices can be so
  converted. But a permutation of columns (e.g., ones) would make this
  impossible.
\item
  True. By linearity:
\end{enumerate}

\[v=c_1u+c_2w\\
A(v)=A(c_1u+c_2w)=c_1Au+c_2Aw\]

\begin{enumerate}
\def\labelenumi{\arabic{enumi}.}
\setcounter{enumi}{39}
\item
  False. If we drop a row from an the RREF of a 1-row matrix, there
  \emph{is} no remaining matrix. Otherwise true, since we will not lose
  any free variables
\item
  True. Systems are only consistent if \(b\) introduces linear
  dependency; that is, the augmented rank equals the rank of the
  original matrix. If not, the \(b\) is not a linear combination of
  \(A\)'s columns and the system is inconsistent.
\item
  True. Full row rank guarantees consistency, and partial column rank
  guarantees free variables.
\item
  True. If two matrices have the same RREF, they are linear combinations
  of each other. Since all linear combinations of the zero vector are
  just the zero vector, they will have exactly the same null space. This
  is \emph{not} true for a particular \(b\); in that case the solutions
  will be the same multiples of each other as are the matrices.
\item
  False. Dropping the RREF column of a fixed variable in a system of
  partial column rank may make it possible to eliminate the previously
  free variable's column. So the new \(R\) is now longer RREF. This
  makes sense, as each column represents a variable.
\item
  True. If two matrices share the same null space, they must have the
  same RREF.
\item
  True. Any zeroes on the diagonals mean partial rank.
\item
  True, by the determinant definition. All nonsingular matrices have
  full rank.
\item
  If \(c_1U+c_2v=w\), then \(u+v+w=u+v+c_1u+c_2v=u+v+c(u+v)\), since the
  sum is itself a linear combination.
\item
  True. If a system is unique for one solution, it is unique for all, so
  if \(Ax=c\) is consistent it must have a unique solution. If \(x\) is
  a linear combination of \(b\) and \(c\), then \(A(b+c)= Ax + Ay\)
\item
  False. Most such matrices will have these two choices of value overlap
  in such a way that they have partial rank.
\end{enumerate}

It seems about a third of such matrices have inverses.

\begin{Shaded}
\begin{Highlighting}[]
\NormalTok{tests <-}\StringTok{ }\KeywordTok{replicate}\NormalTok{(}\DataTypeTok{n =} \DecValTok{500}\NormalTok{, }\KeywordTok{matrix}\NormalTok{(}\KeywordTok{sample}\NormalTok{(}\DecValTok{0}\OperatorTok{:}\DecValTok{1}\NormalTok{, }\DecValTok{9}\NormalTok{, }\DataTypeTok{replace =} \OtherTok{TRUE}\NormalTok{), }\DataTypeTok{nrow =} \DecValTok{3}\NormalTok{), }\DataTypeTok{simplify =} \OtherTok{FALSE}\NormalTok{) }\OperatorTok\StringTok{ }
\StringTok{  }\KeywordTok{suppressWarnings}\NormalTok{(}\KeywordTok{map}\NormalTok{(}\OperatorTok{~}\KeywordTok{try}\NormalTok{(}\KeywordTok{solve}\NormalTok{(.x)))) }\OperatorTok\StringTok{ }
\StringTok{  }\KeywordTok{map_lgl}\NormalTok{(is.matrix) }\OperatorTok\StringTok{ }
\StringTok{  }\KeywordTok{sum}\NormalTok{()}
\NormalTok{tests}
\end{Highlighting}
\end{Shaded}

\begin{verbatim}
## [1] 500
\end{verbatim}

\end{document}
